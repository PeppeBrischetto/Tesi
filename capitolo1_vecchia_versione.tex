Quando nel \textcolor{red}{1998} l'esperimento Super-Kamiokande ha per la prima volta osservato le oscillazioni del neutrino\cite{fukuda:prl98}, si è dimostrato inequivocabilmente che tale particella possiede una massa. 
%Tuttavia, dal momento che tale fenomeno è funzione della differenza dei quadrati delle masse, sono necessarie altre tipologie di esperimenti per accedere alla scala di massa assoluta.
Tuttavia, essendo la probabilità di oscillazione funzione della differenza dei quadrati delle masse, tale fenomeno non permette di conoscere la scala di massa assoluta. 
Altre tipologie di esperimenti sono, dunque, necessarie per accedere al valore della massa del neutrino.
Fra i processi capaci di fornire questa informazione, particolare importanza assume il doppio decadimento beta senza l'emissione di neutrini (\doppiobeta), poiché la sua osservazione permetterebbe di chiarire in modo incontrovertibile se il neutrino è una particella di Dirac o di Majorana.
Inoltre, dal momento che tale processo viola la conservazione del numero leptonico, esso costituisce uno degli esempi più rilevanti di fisica oltre il Modello Standard (MS).
Per queste ragioni il \doppiobeta{} ha attratto a sè grande interesse da parte della comunità scientifica e nell'ultimo decennio innumerevoli esperimenti sono nati in tutto il mondo per osservarlo per la prima volta.
%come testimoniano gli innumerevoli esperimenti volti a misurarne il tempo di dimezzamento. 
%
%
%Il grande interesse della comunità scientifica sul \doppiobeta{} è testimoniato dagli innumerevoli esperimenti che in tutto il mondo provano a misurarne il tempo di dimezzamento.
% OPPURE POTREI SCRIVERE
%Questo processo ha attratto grande interesse da parte della comunità scientifica, come testimoniano gli innumerevoli esperimento che in tutto il mondo provano a misurarne il tempo di dimezzamento.
%
%
%
%Come evidenziato nella Sezione~\ref{sez:progetto_numen}, nell'espressione della probabilità di decadimento del \doppiobeta{} è presente un termine legato alla transizione del nucleo atomico dallo stato iniziale a quello finale. L'accesso per via sperimentale a tale termine è l'obiettivo principale del progetto NUMEN\cite{cappuzzello:epja18} (NUclear Matrix Elements for Neutrinoless double beta decay).  
%
%Come evidenziato nel seguito di questo capitolo, noto il tempo di dimezzamento del \doppiobeta{}, la deduzione della massa del neutrino è subordinata alla conoscenza dell'elemento di matrice che esprime la transizione del nucleo atomico dallo stato iniziale a quello finale. 
%
%L'accesso per via sperimentale a tale elemento di matrice è l'obiettivo principale del progetto NUMEN\cite{cappuzzello:epja18} (NUclear Matrix Elements for Neutrinoless double beta decay).
%
Dal momento che il \doppiobeta{} prevede transizioni fra nuclei atomici, una sua completa descrizione non può prescindere dalla struttura nucleare; in particolare, come mostrato nella~\ref{eq:rate_doppio_beta}, il tempo di dimezzamento dipende dall'elemento di matrice nucleare del processo.
%che esprime la transizione del nucleo dallo stato iniziale a quello finale. 
Il progetto NUMEN\cite{cappuzzello:epja18} (NUclear Matrix Elements for Neutrinoless double beta decay) ha come obiettivo principale l'accesso per via sperimentale a tale elemento di matrice.
%, come verrà spiegato nel seguito di questo capitolo.
\vspace{1cm}
 
In questo capitolo vengono presentate le principali caratteristiche del \doppiobeta{}, spiegando le motivazioni che hanno portato alla nascita del progetto NUMEN, e vengono descritte le ambiziose sfide scientifiche e tecnologiche che tale progetto intende affrontare.


%Il presente lavoro di tesi si colloca all'interno del progetto NUMEN\cite{cappuzzello:epja18} (NUclear Matrix Elements for Neutrinoless double beta decay), il quale propone un nuovo metodo per estrarre informazioni basate sui dati sperimentali sugli elementi di matrice nucleare che entrano in gioco nell'espressione del rate di dimezzamento del doppio decadimento beta senza neutrini (\doppiobeta). 
%Il \doppiobeta{} costituisce una delle aree di interesse più importanti della fisica contemporanea, come testimoniano gli innumerevoli esperimenti che nel mondo mirano alla sua scoperta.
%Come testimoniano gli innumerevoli esperimenti che nel mondo mirano alla sua scoperta, il \doppiobeta{} costituisce una delle aree di interesse più importanti della fisica contemporanea, dal momento che diverse questioni aperte del Modello Standard potrebbero trovare risposta nel caso in cui venisse osservato.


\section{\iflanguage{italian}{Il doppio decadimento beta senza neutrini}{The neutrinoless double beta decay}} \label{sez:doppio_beta_senza_neutrini}

L'idea del doppio decadimento beta fu per la prima volta suggerita da Maria Goeppert-Mayer nel 1935 in un articolo in cui si calcolava la probabilità di emissione simultanea di due elettroni e due \textcolor{red}{anti-neutrini (nell'articolo lei dice neutrini)} come un effetto del secondo ordine della teoria di Fermi del decadimento beta\cite{goeppert-mayer:pr35}. 
%Tale processo, oggi noto come doppio decadimento beta con due neutrini ($ 2\nu\beta\beta $), è contemplato all'interno del MS come un effetto del secondo ordine del decadimento beta.
Tale processo, oggi noto come doppio decadimento beta con due neutrini ($ 2\nu\beta\beta $), è contemplato all'interno del MS ed è stato osservato in undici isotopi, diventando il più raro e lento fenomeno naturale conosciuto.
 
%Il doppio decadimento beta con due neutrini ($ 2\nu\beta\beta $) è previsto all'interno del MS come un effetto del secondo ordine


%Il \doppiobeta{} è invece un processo proibito dal MS ed è possibile soltanto se il neutrino possiede una massa e coincide con la propria antiparticella, ovvero ....
Il \doppiobeta{} fu per la prima volta proposto da Furry nel 1939\cite{furry:pr39}, a seguito di un articolo di Majorana del 1937\cite{majorana:nc37} in cui il fisico catanese formulava l'ipotesi che il neutrino coincidesse con la propria antiparticella, ovvero fosse una \emph{particella di Majorana}. 
%soltanto se il neutrino possiede una massa ed è una particella di Majorana il \doppiobeta{} può avvenire.
%Nell'articolo di Furry veniva evidenziato come il \doppiobeta{} avesse un ruolo cruciale per fare luce sulla natura del neutrino; tale fenomeno è, infatti, possibile soltanto se il neutrino possiede una massa ed è una particella di Majorana. 
Nell'articolo di Furry veniva evidenziato il ruolo cruciale del \doppiobeta{} nella chiarificazione della natura del neutrino; il fenomeno in questione è, infatti, possibile soltanto se il neutrino possiede una massa ed è una particella di Majorana. 



%Proposto per la prima volta da Furry nel 1939\cite{furry:pr39}, il \doppiobeta{} è un processo di decadimento che può avvenire in uno dei modi seguenti:
%\begin{IEEEeqnarray}{rll}
%	& (A, Z) \rightarrow (A, Z+2) + 2e^{-}  & \\
%	& (A, Z) \rightarrow (A, Z-2) + 2e^{+}  & 
%\end{IEEEeqnarray}
%In letteratura il primo tipo di decadimento viene solitamente indicato con $\beta^-\beta^-$, mentre il secondo con $\beta^+\beta^+$.
%Proposto per la prima volta da Furry nel 1939\cite{furry:pr39}, il \doppiobeta{} è un processo di decadimento in cui due neutroni (protoni) in un nucleo atomico si trasformano in due protoni (neutroni) emettendo due elettroni (positroni) e nessun anti-neutrino (neutrino).
%Esso è possibile soltanto se il neutrino ha massa e coincide con la propria antiparticella, ovvero se è una particella di Majorana.
Il \doppiobeta{} è un processo di decadimento in cui due neutroni (protoni) in un nucleo atomico si trasformano in due protoni (neutroni) emettendo due elettroni (positroni) e nessun anti-neutrino (neutrino).
%Dal momento che vengono prodotti due elettroni, la conservazione del numero leptonico viene violata di due unità, rendendo il processo proibito secondo il~MS.
La creazione di due leptoni senza la presenza della corrispondente componente antileptonica implica che la conservazione del numero leptonico venga violata di due unità, rendendo il processo proibito secondo il MS. 
Sebbene fino ad oggi tale violazione non sia mai stata osservata, le teorie che descrivono l'unificazione dell'interazione elettrodebole e quella forte (Grand Unification Theories, GUTs) sono concordi nell'affermare che, ad energie dell'ordine di $10^{15}$ GeV, il numero leptonico cessa di essere un buon numero quantico\cite{pirro:epja06}. 
Ciò significa che il \doppiobeta{} potrebbe aprire la via verso una GUT delle interazioni fondamentali e svelare l'origine dell'asimmetria materia-antimateria presente nell'Universo\cite{vergados:ijmpe16}.
\vspace{1cm}

Il rate di dimezzamento $ \left[ T_{1/2} \right]^{-1} $ del processo può essere espresso come il prodotto di tre fattori, ovvero
\begin{equation} \label{eq:rate_doppio_beta}
	\left[ T_{1/2} \right]^{-1} \; = \; G^{0 \nu} \: \left| M^{0 \nu} \right|^2 \: \left| f ( m_i, U_{ei}, \xi_i ) \right|^2 
\end{equation}
laddove $G^{0 \nu}$ è il fattore cinematico di spazio delle fasi dei due elettroni emessi; $ f ( m_i, U_{ei}, \xi_i ) $ è un termine contenente una combinazione delle masse $m_i$ delle tre specie di neutrini, dei coefficienti di mixing $U_{ei}$ della matrice PMNS e delle fasi di Majorana $\xi_i$; $M^{0 \nu}$ rappresenta l'ampiezza di probabilità di transizione del nucleo dallo stato iniziale $\phi_i$ a quello finale $\phi_f$, ossia
\begin{equation}
	M^{0 \nu} = \bra{\phi_f} \hat{O}^{0 \nu \beta \beta} \ket{\phi_f} 
\end{equation}
in cui $\hat{O}^{0 \nu \beta \beta}$ è l'operatore che descrive il \doppiobeta{}. 
Ad oggi i numerosi esperimenti che tentano di misurare il tempo di dimezzamento del processo sono stati in grado di fornire soltanto dei limiti inferiori; i più recenti risultati affermano che, al 90\% di livello di confidenza, $T_{1/2}$ deve essere maggiore di $8.0 \cdot 10^{25}$~yr nel caso del \ce{^{76}Ge}\cite{agostini:prl18}, e di $1.1 \cdot 10^{26}$~yr nel caso del \ce{^{136}Xe}\cite{gando:prl16}. Tali valori corrispondono ad un limite superiore per la massa del neutrino compreso tra 120 -- 260~meV nel primo caso e tra 50 -- 160~meV nel secondo.




La quantità $M^{0 \nu}$, nota in letteratura come \emph{elemento di matrice nucleare} (\emph{Nuclear Matrix Element}, NME), viene attualmente valutata attraverso avanzati metodi di calcolo, come ad esempio la Quasi-particle Random Phase Approximation (QRPA), il Large-scale Shell Model, l'Interacting Boson Model (IBM), l'Energy Density Functional (EDF) e i calcoli Ab-initio. I vari metodi differiscono essenzialmente per il model space adottato, proponendo schemi di troncamento diversi a seconda dei gradi di libertà considerati rilevanti. 
Sebbene accurate informazioni provenienti da esperimenti di singolo scambio di carica (Single Charge Exchange, SCE), reazioni di transfer e cattura elettronica siano state utilizzate per porre dei vincoli ai calcoli teorici, le differenze tra i modelli sono ancora piuttosto grandi, tanto da osservare in alcuni casi discrepanze di un fattore due o tre, come si può evincere dalla Figura~\ref{fig:NME}. 

\begin{figure} [!t]
	\centering
	\includegraphics[scale=0.4]{Grafici/NME.png}
	\caption{I valori dei NMEs in funzione del numero di neutroni calcolati secondo i modelli IBM-2 \cite{barea:prc13}, QRPA-T\"{u}\cite{simkovic:prc13} e ISM\cite{menendez:npa08}. Figura tratta da~\cite{barea:prc15}.} \label{fig:NME}
\end{figure}




%In questo scenario appare evidente la necessità di dedurre dai dati sperimentali nuove informazioni, così da imporre limiti più stringenti ai modelli.  
In questo scenario appare evidente la necessità di imporre limiti più stringenti ai modelli teorici, deducendo dai dati sperimentali nuove informazioni. 
%infatti, nonostante i NMEs non siano direttamente misurabili, sotto opportune condizioni e grazie a modelli teorici appropriati è possibile desumerne il valore tramite misure sperimentali di sezioni d'urto assolute.
Grazie alla loro somiglianza con le transizioni nucleari del \doppiobeta{}, le reazioni di \emph{doppio scambio di carica} (Double Charge Exchange, DCE), ovvero le reazioni in cui la carica nucleare cambia di due unità lasciando invariato il numero di massa, si configurano come un potente strumento d'indagine. 
%A causa della bassa sezione d'urto di tali processi, al fine di identificare le reazioni di DCE è essenziale la misura gli spettri energetici con grande risoluzione e le sezioni d'urto assolute ad angoli prossimi a zero.
A causa della bassa sezione d'urto di tali processi, al fine di identificare le reazioni di DCE è essenziale misurare con grande risoluzione e accuratezza sia gli spettri energetici sia le sezioni d'urto assolute ad angoli prossimi a zero. Inoltre, risulta necessario misurare anche gli altri canali di reazione, in modo da  identificare e quantificare i processi di transfer di nucleoni multi-step che concorrono al meccanismo diretto. Questi contributi possono essere minimizzati grazie ad una scelta opportuna del sistema proiettile-target e dell'energia incidente.


\section{\iflanguage{italian}{Il progetto NUMEN}{The NUMEN project}} \label{sez:progetto_numen}

Il progetto NUMEN propone un nuovo metodo per estrarre informazioni basate sui dati sperimentali (\textcolor{red}{meglio scrivere data-driven?}) sui NMEs che entrano in gioco nel calcolo del rate di dimezzamento del \doppiobeta{}. 
%utilizzando misure accurate di sezioni d'urto di reazioni di DCE indotte da ioni pesanti. 
Per raggiungere tale scopo si intende misurare accuratamente le sezioni d'urto di reazioni di DCE indotte da ioni pesanti, esplorando a diverse energie del fascio incidente \emph{tutti} gli isotopi coinvolti negli esperimenti presenti e futuri sul \doppiobeta{}.
%In particolare, è importante verificare se le sezioni d'urto misurate del DCE sono legate ai NMEs del \doppiobeta{} come una funzione lentamente variabile dell'energia del proiettile e della massa del sistema.%cioè tipo $M^{DCE} \propto f(E_p, A, M^{0 \nu})$
%In tal caso, sarebbe possibile accedere agli elementi di matrice del \doppiobeta{} tramite misure di sezioni d'urto sperimentali. Dal punto di vista teorico, è necessario descrivere accuratamente il meccanismo di reazione, che deve essere fattorizzato in una parte di reazione ed una di struttura nucleare, con quest'ultima a sua volta fattorizzata nel termine del proiettile e in quello del bersaglio.

Il principale, e più ambizioso, obiettivo di NUMEN è l'accesso ai NMEs del \doppiobeta{} attraverso un approccio sperimentale. A tal fine bisogna verificare se gli elementi di matrice del DCE sono legati ai NMEs del \doppiobeta{} come una funzione lentamente variabile dell'energia del proiettile e della massa del sistema.
Qualora questa ipotesi fosse verificata, allora sarebbe possibile dedurre i NMEs del \doppiobeta{} a partire da misure di sezioni d'urto.
% ******* Prima avevo scritto questo:
%Se i risultati sperimentali confermassero che gli elementi di matrice del DCE sono legate ai NMEs del \doppiobeta{} come una funzione lentamente variabile dell'energia del proiettile e della massa del sistema, allora sarebbe possibile dedurre questi ultimi a partire da misure di sezioni d'urto. 
Ciò richiede che il meccanismo di reazione possa essere descritto come il prodotto di un fattore dovuto alla mera reazione e di uno relativo alla struttura nucleare, con quest'ultimo a sua volta fattorizzato in un termine del proiettile e in uno del bersaglio.
%Tale approccio si è dimostrato valido nel caso delle reazioni di singolo scambio di carica (vv. articolo Taddeucci 1987). 
Dunque, lo sviluppo di una teoria microscopica coerente della reazione di DCE è parte indispensabile del progetto. 
Dal punto di vista sperimentale, la verifica della validità di questa ipotesi richiede la costruzione di una sistematica di dati, affrontando le sfide connesse alla ricerca di fenomeni tanto rari, come la bassa sezione d'urto, la grande quantità di background, la necessità di alta risoluzione e sensibilità. 

Altro importante obiettivo di NUMEN consiste nella validazione delle teorie di struttura nucleare che si occupano di calcolare i NMEs del \doppiobeta{};
% ****** Prima avevo scritto:
%; infatti, poiché gli elementi di matrice del DCE e quelli del \doppiobeta{} contengono le stesse funzioni d'onda iniziali e finali e operatori di transizione con struttura simile, la misura di sezioni d'urto assolute può sondare la bontà dei model space adottati dai diversi metodi di calcolo.
infatti, gli elementi di matrice del DCE e quelli del \doppiobeta{} contengono le stesse funzioni d'onda iniziali e finali e operatori di transizione con struttura simile. Se scegliendo un determinato modello di struttura nucleare (con i relativi troncamenti alla funzione d'onda many-body) si trova un buon accordo con i dati sperimentali sulla sezione d'urto del DCE, allora quello stesso model space deve descrivere bene le funzioni d'onda del \doppiobeta.
% Dalla tesi di Ale: "Validare con i dati sperimentali l’applicazione di certi tagli sullo spazio di modello usato nell’analisi dei dati di DCE serve a validare la scelta dello stesso spazio di modello quando l’operatore non è più quello del doppio scambio di carica ma quello del decadimento 0νββ. In questo senso risulta essenziale avere il pieno controllo sulla componente di reazione della sezione d’urto."
Quindi, una volta scelte le funzioni d'onda dal confronto con le sezioni d'urto del DCE, le stesse possono essere impiegate per i NMEs del \doppiobeta{}. 

%Infine, NUMEN potrebbe fornire informazioni sulla sensibilità necessaria per la misura del tempo di dimezzamento del \doppiobeta{} a seconda dell'isotopo utilizzato. 
%Infine, NUMEN potrebbe fornire informazioni importanti sui diversi isotopi utilizzati nella ricerca del \doppiobeta{}, perché, facendo il rapporto delle sezioni d'urto assolute misurate negli esperimenti di DCE, si ottiene una stima di quanto il processo sia probabile indipendentemente dal modello adottato. Questa procedura, che consente di ridurre la presenza di eventuali errori sistematici poiché nel rapporto i due contributi si compensano, potrebbe 
Infine, NUMEN potrebbe fornire informazioni importanti sui diversi isotopi utilizzati nella ricerca del \doppiobeta{}, perché il rapporto delle sezioni d'urto assolute misurate negli esperimenti di DCE offre una stima di quanto il processo sia probabile indipendentemente dal modello assunto. 
%Questa procedura, che consente di ridurre la presenza di eventuali errori sistematici poiché nel rapporto i due contributi si compensano, potrebbe permettere di confrontare
Questa procedura consente di ridurre la presenza di eventuali errori sistematici poiché nel rapporto i due contributi si compensano.
Tale tipologia di analisi potrebbe avere un grande impatto sui futuri esperimenti sul \doppiobeta{}, in quanto potrebbe dare indicazioni su quale isotopo può essere il miglior candidato alla scoperta del processo e sulla sensibilità necessaria per la sua osservazione. 





\subsection{\iflanguage{italian}{Reazioni di DCE e \doppiobeta}{DCE reactions and \doppiobeta}}

Sebbene le reazioni di DCE e il \doppiobeta{} siano mediati da interazioni differenti, ci sono diverse importanti similarità fra loro.
In primo luogo, gli stati nucleari inziali e finali del DCE coincidono con quelli del \doppiobeta{}, in quanto in entrambi i processi avviene la trasformazione di due neutroni (protoni) in due protoni (netroni). 
Un'altra significativa somiglianza riguarda gli operatori di transizione, i quali in tutte e due i casi contengono le componenti a corto range di Fermi, Gamow-Teller e tensoriale di rango-2, con un peso relativo che nelle reazioni di DCE dipende dall'energia incidente. 
Inoltre, in entrambi i processi nel canale intermedio virtuale l'impulso lineare è molto grande, dell'ordine di 100~MeV/c\cite{barea:prl12}. Questo è un aspetto cruciale, poiché significa che sia le reazioni di DCE sia il \doppiobeta{} sondano stati ad alto impulso della funzione d'onda nucleare, mentre altri processi non ne sono in grado\cite{puppe:prc11}.



\section{\iflanguage{italian}{L'upgrade dell'apparato sperimentale}{Upgrade of the experimental set-up}}

Da quanto appena detto è emerso che le reazioni di DCE possono essere uno strumento utile per la comprensione del \doppiobeta{}, in quanto permettono di studiare un fenomeno estremamente raro attraverso un meccanismo che, essendo guidato dall'interazione forte, possiede dei tempi caratteristici molto più brevi. 
Tuttavia, sebbene più probabili, le reazioni di DCE presentano delle sezioni d'urto molto basse, tipicamente di alcune decine di nb.
Di conseguenza, per accumulare una statistica sufficiente, possono essere necessari lunghi tempi di raccolta dei dati o fasci di intensità molto elevata.



\cleardoublepage

\section{\iflanguage{italian}{Le fasi del progetto}{The phases of the project}}

Il progetto NUMEN è articolato in quattro fasi. ballfbaldfb fblakflk badlkfbakdf bdlfkadl b blabdf ka baklba kfb