
All'interno dell'intensa campagna di studi in corso per verificare se le soluzioni previste per il progetto NUMEN possano garantire le prestazioni necessarie, è stato svolto ad Aprile 2019 un test beam sui primi due prototipi di telescopi basati sulla tecnologia SiC-CsI.
Tale test è stato svolto ai Laboratori Nazionali del Sud (LNS) utilizzando un fascio di particelle di \ce{^{20}Ne} a 20 AMeV, prodotto dal Ciclotrone Superconduttore K800.
Al fine di favorire la generazione degli ioni tipicamente di interesse per NUMEN, ovvero Ossigeno, Fluoro e Neon, si è scelto di utilizzare un bersaglio di \ce{^{12}C} da 400~$\mu$g/cm\ap{2}.

In questo capitolo vengono presentati i risultati ottenuti dall'analisi dei dati raccolti durante tale test e viene mostrato il confronto tra questi e i dati prodotti attraverso la simulazione implementata per questo lavoro di tesi. 

\section{\iflanguage{italian}{I dati del telescopio SiC-CsI}{Data of SiC-CsI telescope}}


Gli eiettili prodotti nell'interazione fra proiettile e bersaglio, dopo aver attraversato il quadrupolo e il dipolo, giungevano al rivelatore di piano focale (Focal Plane Detector, FPD) di MAGNEX, dove erano stati posti i due telescopi, chiamati Tel~A e Tel~B. 
Come descritto nella Sezione~\ref{sez:test}, insieme ai due telescopi, erano stati montati quattro dei rivelatori al silicio attualmente in uso, i quali servivano da riferimento e da modello per la Particle IDentification (PID).

Poiché all'energia utilizzata nel test i prodotti di reazione non erano in grado di superare il substrato epitassiale del Tel~B, da tale telescopio era possibile estrarre soltanto il segnale sulla perdita di energia, non permettendo di riprodurre le correlazioni $\Delta E - E_{resid}$ utili per l'identificazione degli ioni.
Di conseguenza, dal momento che questo lavoro aveva lo scopo di studiare le performance di PID di tale sistema, sono stati analizzati soltanto i dati relativi al Tel~A.


\subsection{\iflanguage{italian}{Le matrici $\Delta E_{SiC} - E_{CsI}$}{$\Delta E_{SiC} - E_{CsI}$ matrices}}

Una delle matrici $\Delta E_{SiC} - E_{CsI}$ registrate durante il test è riportata in Figura~


\section{\iflanguage{italian}{Analisi dei dati del test beam}{Analysis of test beam data}}

\subsection{\iflanguage{italian}{Identificazione dell'\ce{^{16}O}}{Identification of \ce{^{16}O}}}


\section{\iflanguage{italian}{Confronto fra i dati del test e la simulazione \geant}{Comparison between test data and \geant simulation}}
