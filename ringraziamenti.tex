

A conclusione di questo lavoro, che segna la fine di un lungo viaggio ma apre le porte ad una nuova avventura, desidero ringraziare tutte le persone che mi sono state accanto e che, in modi diversi, hanno contribuito a farmi raggiungere questo traguardo.
%Quando si arriva alla mia veneranda età, le persone da ringraziare diventano tante, mentre la memoria 

In primo luogo, ringrazio il Prof. Francesco Cappuzzello, relatore di questa tesi, che con i suoi preziosi consigli mi ha guidato magistralmente lungo questo percorso.
Lei rappresenta, a mio parere, il modello di professore ideale, sempre pronto ad ascoltare gli studenti e a chiarire i loro dubbi.
Dal profondo del mio cuore ringrazio il Dott. Luciano Pandola, correlatore di questo lavoro, per avermi trasmesso tantissime conoscenze e per avermi sempre aiutato, con una gentilezza impareggiabile.
Le Sue parole di incoraggiamento mi hanno fatto sentire in grado di poter fare questo mestiere; inoltre, condividiamo la stessa passione per il punto e virgola.
Uno speciale ringraziamento va alla Dott.ssa Manuela Cavallaro, per essere stata sempre pronta ad ascoltare i miei progressi e i miei dubbi, regalandomi preziosi suggerimenti e chiarimenti.
Con immensa gratitudine ringrazio la Dott.ssa Diana Carbone, perché tutte le volte che ho avuto bisogno del Suo aiuto c'è sempre stata.
Lei mi ha insegnato tanto e mi ha 
Ringrazio con grande affetto Akis e Vasilis, che con incommensurabile gentilezza e disponibilità mi hanno aiutato a completare questo lavoro, sedendosi accanto a me e mettendo da parte i loro impegni pur di aiutarmi.
Rivolgo un sincero ringraziamento a Domenico e Turi, per essere stati sempre disponibili a rispondere alle mie domande e a chiarire i miei dubbi.


Ringrazio i miei genitori, Michelangelo e Concetta, che mi hanno cresciuto con tanto amore e mi hanno sempre sostenuto e incoraggiato.
Non ve lo dico spesso, ma mi voglio tanto bene.
Se oggi posso festeggiare questo traguardo è merito Vostro.
Ringrazio mio fratello Alberto, con cui ho condiviso tante esperienze nel corso della mia vita e che mi ha regalato tante risate.
I fratelli non li puoi scegliere: io sono stato fortunato.
Grazie per esserci stato nei momenti difficili (o, come dici tu, di \emph{burn out}): non lo dimenticherò mai.

Ringrazio il mio amore, Carla, che ogni giorno mi fa sentire amato.
Nei momenti difficili, un tuo sguardo o una tua parola mi hanno dato la forza di andare avanti.
Da quel primo bacio sono passati più di sette anni 