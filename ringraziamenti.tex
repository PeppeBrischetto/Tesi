

A conclusione di questo lavoro, che segna la fine di un lungo viaggio ma apre le porte ad una nuova avventura, desidero ringraziare tutte le persone che mi sono state accanto e che, in modi diversi, hanno contribuito a farmi raggiungere questo traguardo.
%Quando si arriva alla mia veneranda età, le persone da ringraziare diventano tante, mentre la memoria 
Perdonatemi se questi ringraziamenti sono un po' lunghi, ma sentivo il bisogno di ringraziarvi tutti.



In primo luogo, ringrazio il Prof. Francesco Cappuzzello, relatore di questa tesi, che con i suoi preziosi consigli mi ha guidato magistralmente lungo questo percorso.
Lei rappresenta, a mio parere, il modello di professore ideale, sempre pronto ad ascoltare gli studenti e a chiarire i loro dubbi.
Dal profondo del mio cuore ringrazio il Dott. Luciano Pandola, correlatore di questo lavoro, per avermi trasmesso tantissime conoscenze e per avermi sempre aiutato, con una gentilezza impareggiabile.
Le Sue parole di incoraggiamento mi hanno fatto sentire in grado di poter fare questo mestiere; inoltre, condividiamo la stessa passione per il punto e virgola.
Uno speciale ringraziamento va alla Dott.ssa Manuela Cavallaro, per essere stata sempre pronta ad ascoltare i miei progressi e i miei dubbi, regalandomi preziosi suggerimenti e chiarimenti.
Con immensa gratitudine ringrazio la Dott.ssa Diana Carbone, perché tutte le volte che ho avuto bisogno del Suo aiuto c'è sempre stata.
Per tutto quello che mi ha insegnato e per tutti i preziosi consigli, Le dico grazie.
%Lei mi ha insegnato tanto, sin dal mio primo shift in un esperimento
Ringrazio con grande affetto Akis e Vasilis, che con incommensurabile gentilezza e disponibilità mi hanno aiutato a completare questo lavoro, sedendosi accanto a me e mettendo da parte i loro impegni pur di aiutarmi.
Rivolgo un sincero ringraziamento a Domenico e Turi, per essere stati sempre disponibili a rispondere alle mie domande e a chiarire i miei dubbi.

%\newpage
Ringrazio i miei genitori, Michelangelo e Concetta, che mi hanno cresciuto con tanto amore e mi hanno sempre sostenuto e incoraggiato.
Non ve lo dico spesso, ma mi voglio tanto bene.
Se oggi posso festeggiare questo traguardo è merito Vostro.
Ringrazio mio fratello Alberto, con cui ho condiviso tante esperienze nel corso della mia vita e che mi ha regalato tante risate.
I fratelli non li puoi scegliere: io sono stato fortunato.
Grazie per esserci stato nei momenti difficili (o, come dici tu, di \emph{burn out}): non lo dimenticherò mai.

Ringrazio il mio amore, Carla, che ogni giorno mi fa sentire amato.
Nei momenti difficili, un tuo sguardo o una tua parola mi hanno dato la forza di andare avanti: sei stata la luce che mi ha guidato attraverso l'oscurità.
È difficile esprimere in poche righe tutto l'amore che provo per te, quindi ti dirò semplicemente che ti amo, con tutto il mio cuore.
%Ti~amo.
%Da quel primo bacio sono passati più di sette anni 


Ringrazio la Sig.ra~Michelia, che mi ha accolto a casa sua e mi ha sempre trattato come un figlio, condividendo le gioie e i momenti difficili.
Ringrazio Alfio, che nei miei confronti ha sempre dimostrato tanta simpatia e gentilezza.
Un ringraziamento speciale va al Sig.~Silvano, con cui abbiamo intrapreso tante discussioni interessanti e stimolanti sulla fisica e sulla filosofia, e alla Sig.ra~Pippa, che mi vuole bene come ad un nipote.


Dedico un grande e affettuoso ringraziamento ai miei colleghi, che mi hanno accompagnato lungo questo cammino.
In particolare, ringrazio Alessandro, con cui ho avuto tante interessanti conversazioni, che mi hanno spesso aiutato a comprendere meglio la fisica.
Ringrazio Maria Lucia, sempre pronta a sostenermi e ad ascoltarmi: i tuoi consigli sono stati molto preziosi.
Non potrò mai dimenticare che tu ed Alessandro, in uno dei momenti più difficili, vi siete seduti accanto a me e mi avete aiutato a completare il Capitolo~4.
Ringrazio Irene, che ogni mattina mi ha incoraggiato dicendomi che mi trovava un po' più riposato e con meno occhiaie: grazie per tutte le volte che ti ho chiesto aiuto e tu ci sei sempre stata.
Ringrazio Laura, che mi è stata accanto nei momenti di disperazione, facendo anche uno shift al posto mio.
Non ho parole per esprimerti la mia gratitudine: mi hai ascoltato e incoraggiato come una vera amica.
Ringrazio Simone, che tante volte mi ha aiutato a risolvere i problemi incontrati in questi mesi, e lo ha fatto sempre con tanta gentilezza e disponibilità.
Ringrazio Giulia, Giuseppe e Filippo, perché ogni giorno mi hanno regalato delle parole di sostegno.
Ringrazio i colleghi di vecchia data, Giovanna, Leonard e Stefano, che da tanti anni mi onorano della loro amicizia.

Ringrazio gli amici di \emph{Lenticchio}: Roberta, Fabrizio, Stefania, Zuzzo, Marcello e Pippo. 



