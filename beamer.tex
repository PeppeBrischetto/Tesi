\documentclass[pt12]{beamer}
%\documentclass[pt12,externalviewer]{beamer}
\usepackage[italian]{babel}
\usepackage[utf8]{inputenc}
\usepackage[T1]{fontenc}
\usepackage{multirow}
\usepackage{booktabs} % Allows the use of \toprule, \midrule and \bottomrule in tables
\usepackage{graphicx} %Allows including images

\usepackage{verbatim}
\usepackage{amssymb}
\usepackage{amsfonts}
\usepackage{amsmath}
\usepackage{setspace}
\usepackage{siunitx}
\usepackage[version=4]{mhchem}
\usepackage{mathrsfs}
\usepackage{braket}
\usepackage{ragged2e}
%\usepackage{rotating}
%\usepackage{amsmath}
%\usepackage{tikz}
%\usepackage{beamergraphics}
%\usepackage[italian]{babel}
%\usepackage[latin1]{inputenc}
%\usepackage[T1]{fontenc}

%\usepackage[absolute,overlay]{textpos}
%\usepackage{pdfcolparallel}

%\usepackage{tcolorbox}
%\tcbuselibrary{fitting}

\definecolor{CTgreen}{HTML}{28794e} % Catania green (primary)
\definecolor{CTblue}{HTML}{28794e} % Catania green (primary)

\mode<presentation>
{
  \usetheme{Warsaw} 
%  \usetheme{Madrid}
%  \usetheme{Montpellier}
%  \usetheme{Marburg} 
  \usecolortheme[named=CTgreen]{structure}
  \setbeamercolor{alerted text}{fg=CTgreen}
  \setbeamercovered{transparent}
  \setbeamertemplate{section in toc}[ball unnumbered]
}

%\setbeamertemplate{footline}{\hfill\insertframenumber/\inserttotalframenumber} 

\expandafter\def\expandafter\insertshorttitle\expandafter{%
  \insertshorttitle\hfill%
  \insertframenumber\,/\,\inserttotalframenumber}

\newcommand{\backupbegin}{
   \newcounter{framenumberappendix}
   \setcounter{framenumberappendix}{\value{framenumber}}
}
\newcommand{\backupend}{
   \addtocounter{framenumberappendix}{-\value{framenumber}}
   \addtocounter{framenumber}{\value{framenumberappendix}} 
}

\newcommand{\refer}[1]{%
   \begin{flushright}
      {\alert{\tiny #1}}
   \end{flushright}}
  
\newcommand{\lrefer}[1]{%
   \begin{flushleft}
      {\alert{\tiny #1}}
   \end{flushleft}}
  
\newcommand{\param}[1]{%
   \begin{flushright}
      {\small #1}
   \end{flushright}
   \vspace{-1.5\baselineskip}
}

\newcommand{\sech}{\mathop{\rm sech}\nolimits}
\newcommand{\sgn}{\mathop{\rm sgn}\nolimits}
\newcommand{\etal}{{\em et al.}}


\title[Simulazioni Monte Carlo per il progetto NUMEN]{Simulazioni Monte Carlo di un sistema di rivelazione per ioni pesanti basato sulla tecnologia SiC-CsI per il progetto NUMEN}

\author[Giuseppe Antonio Brischetto]{\large{Giuseppe Antonio Brischetto}}

\institute[DFA.UniCT]{
\begin{minipage}[c]{1.5truecm}
\includegraphics[width=\textwidth]{logo_ellipse}
\end{minipage}
\begin{minipage}[c]{4.7truecm}
\begin{flushleft}
\begin{sl}
Dipartimento di Fisica e Astronomia\\ 
``Ettore Majorana''
\end{sl}
\end{flushleft}
\end{minipage}
\begin{minipage}[c]{2truecm}
\includegraphics[width=\textwidth]{logo_unict_orizzontale}
\end{minipage}

\vspace*{20pt}
\begin{columns}
	\begin{column}{0.6\textwidth} \hspace*{30pt}
		%\centering
		{\normalsize Relatore:\\ \hspace*{30pt}
		Chiar.mo~Prof.~F.~Cappuzzello} 
	\end{column}

	\begin{column}{0.6\textwidth} \hspace*{60pt}
		%\centering
%		\hspace{0.2truecm}
		{\normalsize Correlatore:\\ \hspace*{60pt}
		Dott.~L.~Pandola}
	\end{column}
\end{columns}}




\date{22 Ottobre 2019}

\begin{document}

\begin{frame}[plain]
\titlepage
\end{frame}



\begin{frame}[label=outline]
\frametitle{Outline}
\tableofcontents[pausesections]
\end{frame}

%\include{sslide}

%\clearpage
%\phantomsection
%\addcontentsline{toc}{chapter}{Introduzione}
\section{Il contesto scientifico}
%\subsection*{Il doppio decadimento beta senza neutrini}
%\begin{frame}
%	\frametitle{Problemi aperti sul neutrino}
%	Oscillazioni di sapore del neutrino $\rightarrow$ il neutrino è massivo
%\end{frame}


\subsection*{Il doppio decadimento beta senza neutrini}
\begin{frame}
	\frametitle{Il doppio decadimento beta senza neutrini}
	%\centering
	 Il neutrino oscilla $\longrightarrow $ ha una \textcolor{red}{massa}
	
	\begin{block}
	 Il neutrino oscilla $\longrightarrow $ ha una \textcolor{red}{massa}
	\end{block}

	 $$\ce{^A_ZA_{N}}\rightarrow\ce{^A_{Z\mp2}B_{N\pm2}}$$
	 \begin{columns}
	 	\begin{column}{0.5\textwidth}
%			$[T^{0 \nu}_{1/2}]^{-1}  =  G^{0 \nu}  | M^{0 \nu} |^2 | f(m_i, U_{ei}) |^2$
			
			\begin{equation*}
			[T^{0\nu}_{1/2}]^{-1} \: = \: G_{0\nu} \, \left|M_{0\nu}\right|^2 \, \left|f(m_i, U_{ei})\right|^2
			\end{equation*}
			%$$\ce{^A_ZA_{N}}\rightarrow\ce{^A_{Z\mp2}B_{N\pm2}}$$
			%$$[T^{0\nu}_{1/2}]^{-1}=G_{0\nu}\left|M_{0\nu}\right|^2\left|f(m_i, U_{ei}, \xi_i)\right|^2$$
	 	\end{column}
	 \end{columns}
\end{frame}


\subsection*{Le reazioni di doppio scambio di carica}
\begin{frame}
	\frametitle{Le reazioni di doppio scambio di carica}
    \begin{itemize}[<+->]
    	\item ciao
    	\item a tutti
    \end{itemize}
\end{frame}


\subsection*{Il progetto NUMEN}
\begin{frame}
	\frametitle{Il progetto NUMEN}
	\begin{itemize}[<+->]
		\item ciao
		\item a tutti
	\end{itemize}
\end{frame}


%\include{sslide}

\end{document}

