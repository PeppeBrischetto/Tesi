\documentclass[10pt,foldmark,notumble]{leaflet}
\renewcommand*\foldmarkrule{.3mm}
\renewcommand*\foldmarklength{5mm}
\usepackage[italian]{babel}
\usepackage[utf8]{inputenc}
\usepackage[T1]{fontenc}

\usepackage{graphicx}
\usepackage{amsmath}
\usepackage{textcomp}
\usepackage{mathptmx}
\usepackage[scaled=0.9]{helvet}
\usepackage{lipsum}

\usepackage[dvipsnames,usenames]{color}
\definecolor{LIGHTGRAY}{gray}{.9}

%%%%\renewcommand{\descfont}{\normalfont}
%\newcommand\Lpack[1]{\textsf{#1}}
%\newcommand\Lclass[1]{\textsf{#1}}
%\newcommand\Lopt[1]{\texttt{#1}}
%\newcommand\Lprog[1]{\textit{#1}}

\newcommand*\defaultmarker{\textsuperscript\textasteriskcentered}

\newcommand{\doppiobeta}{$ 0\nu\beta\beta$}
\newcommand{\geant}{Geant4}

%\title{\bf Carolina Dynamics Symposium}

%\author{%
%\Large \bf Clemson University
%  Martin Schmoll\\
%  Predrag Puno\v{s}evac
%}
%\date{\bf April 13 -- April 15, 2012 }

\CutLine*{1}% Dotted line without scissors
\CutLine*{6}% Dotted line without scissors
%\CutLine{6}%  Dotted line with scissors

%\AddToBackground{1}{%  Background of a small page
%  \put(0,0){\textcolor{Cerulean}{\rule{\paperwidth}{\paperheight}}}}


%\AddToBackground{1}{%  Background of a small page
%  \put(40,200){\includegraphics[scale=0.04]{numen_logo.jpeg}}}


%\AddToBackground{6}{%  Background of a small page
%  \put(0,0){\textcolor{YellowOrange}{\rule{\paperwidth}{\paperheight}}}}


%\AddToBackground*{2}{% Background of a large page
%  \put(\LenToUnit{.5\paperwidth},\LenToUnit{.5\paperheight}){%
%    \makebox(0,0)[c]{%
%      \resizebox{.9\paperwidth}{!}{\rotatebox{35.26}{%
%        \textsf{\textbf{\textcolor{LIGHTGRAY}{CLEMSON}}}}}}}}

%\AddToBackground*{1}{% Background of a large page
%  \put(\LenToUnit{.66\paperwidth},\LenToUnit{.36\paperheight}){%
%    \makebox(0,0)[c]{%
%      \resizebox{.3\paperwidth}{!}{\rotatebox{0.0}{%
%        \includegraphics{numen_logo.jpeg} }}}}}


\begin{document}

\begin{center}
%\includegraphics[width=16em]{unict_dfa}\\
%\uppercase{Universit\`a degli Studi di Catania}\\
%{\sc dipartimento di fisica e astronomia}\\
%{\sc corso di laurea triennale in fisica}\\
\begin{minipage}[c]{0.45\textwidth}
\begin{flushleft}
\includegraphics[width=0.8\textwidth]{logo_unict_orizzontale}
\end{flushleft}
\end{minipage}
\hfill
\begin{minipage}[c]{0.45\textwidth}
\begin{flushright}
\includegraphics[width=\textwidth]{logo_dfa_orizzontale}
\end{flushright}
\end{minipage}\\
\medskip
{\sc corso di laurea magistrale in fisica}\\
\hbox to \textwidth{\hrulefill}

\vspace{3truecm}

{\sc Giuseppe Antonio Brischetto}

\vfill

\uppercase{\sc Simulazioni Monte Carlo di un sistema di rivelazione per ioni pesanti basato sulla tecnologia SiC-CsI per il progetto NUMEN}

\vfill

\centerline{\hbox to 3.5truecm{\hrulefill}}
{\sc elaborato finale}\\
%{\sc tesi di laurea}\\
\centerline{\hbox to 3.5truecm{\hrulefill}}

\vfill

\begin{minipage}{\textwidth}
\begin{flushright}
\begin{minipage}{0.3\textwidth}
\begin{tabbing}
Chiar.mo \= Prof. P. Pallino \kill
Relatore: \> \\
Chiar.mo \> Prof. F. Cappuzzello \\
\\
\textsc{Correlatore:} \\
\textsc{Dott. L. Pandola}
\end{tabbing}
\end{minipage}
\end{flushright}
\end{minipage}

\vfill

\hbox to \textwidth{\hrulefill}
{\sc anno accademico 2018/2019}

\end{center}

\newpage

\section{Abstract}

%\lipsum[1]
%Questo lavoro di tesi verte sulla realizzazione di un tool di simulazioni Monte Carlo per lo studio di un sistema di rivelazione basato sulla tecnologia SiC-CsI per l'identificazione di ioni pesanti nell'ambito del progetto NUMEN.
%Tale progetto propone ai Laboratori Nazionali del Sud un nuovo metodo per estrarre dai dati sperimentali informazioni sugli elementi di matrici nucleare che entrano in gioco nel tempo di dimezzamento del doppio decadimento beta senza neutrini.
Il progetto NUMEN ha avviato ai Laboratori Nazionali del Sud un'importante ristrutturazione verso alte correnti di fascio delle due principali infrastrutture sperimentali: il Ciclotrone Superconduttore K800 e lo spettrometro magnetico MAGNEX.
In particolare, l'upgrade di MAGNEX prevede l'introduzione di un muro di telescopi $\Delta E - E$ a stato solido, basato sulla tecnologia SiC-CsI, per l'identificazione di ioni pesanti.
Questo lavoro di tesi verte sulla realizzazione di un tool di simulazioni Monte Carlo per lo studio e l'ottimizzazione di tale sistema di rivelazione.
%Il tool è stato sviluppato e implementato integrando i software per la ricostruzione del moto dei prodotti di reazione attraverso lo spettrometro con un'applicazione specifica basata sulla piattaforma \geant.
Il principale obiettivo di questo lavoro consiste nella valutazione delle prestazioni del sistema simulato in termini di capacità di identificazione degli ioni di interesse, rapporto segnale-rumore e sensibilità di misura.
In aggiunta, il tool è stato utilizzato per ottimizzare le condizioni di granularità dei telescopi.



\section{Background}

\lipsum[2-4]

\section{Objectives}

\lipsum[5-7]

\section{Techniques}

\lipsum[8-10]

\section{Results}

\lipsum[11]

\includegraphics[width=0.9\columnwidth]{esempio}

\lipsum[12-13]

\section{Conclusions}

\lipsum[14]

\section{Acknowledgements}

\lipsum[1]

\section{Contact details}

Giuseppe Antonio Brischetto\\ {\tt nome.cognome@studium.unict.it}

\end{document}
