\documentclass[MSc,italian]{dfaunictthesis}
\usepackage{lipsum}
\usepackage[compat=1.0.0]{tikz-feynman}
\usepackage{braket}
\usepackage{mhchem}
\usepackage{textcomp}
%\usepackage{mathtools}
%\usepackage{mathrsfs}
%\usepackage{amsfonts}
%\usepackage{amsmath}
%\usepackage{amssymb}
%\usepackage[retainorgcmds]{IEEEtrantools}
\usepackage[hypcap=true]{caption}
\usepackage[Lenny]{fncychap}
\hypersetup{
%	pdfpagemode={UseOutlines},
%	bookmarksopen,
%	colorlinks,
%	linkcolor=red,
%	anchorcolor=red,
%	citecolor=red,
%	urlcolor=red,
%	%linktocpage=true
%	pdftitle={La condensazione di Bose-Einstein},
%	pdfauthor={Giuseppe Antonio Brischetto}
}


\newcommand{\doppiobeta}{$ 0\nu\beta\beta$}
\usepackage{url}

\begin{document}

\author{Giuseppe Antonio Brischetto}
\title{Simulazioni Monte Carlo di un sistema di rivelazione per ioni pesanti basato sulla tecnologia SiC-CsI per il progetto NUMEN}
\aayear{2018/2019}

\begin{supervisors}
   \supervisor{Chiar.mo}{Prof.}{F. Cappuzzello}
   \supervisor{}{Dr.}{L. Pandola}
\end{supervisors}

%\phdname{physics} % default
%\phdname{science of materials}
%\phdname{complex systems}

%\maketitlepage

%\tableofcontents
\thispagestyle{empty}
% Le prossime righe servono per aggiungere l'indice alla barra dei segnalibri nel pdf.
\cleardoublepage
\pdfbookmark[1]{Indice}{Indice}
\tableofcontents
%\thispagestyle{empty}
%\clearpage
%\thispagestyle{empty}

% Le prossime due righe servono per sistemare il collegamento ipertestuale nell'indice. Si è usato \cleardoublepage perché la classe è book, mentre per article si usa \clearpage. Comunque vedere ArteLateX.pdf per ogni evenienza
%\cleardoublepage
%\phantomsection
%\chapter*{\iflanguage{italian}{Introduzione}{Introduction}}
%\setcounter{page}{1}
%\addcontentsline{toc}{chapter}{\iflanguage{italian}{Introduzione}{Introduction}}

%\lipsum[20]

\chapter{\iflanguage{italian}{Il contesto scientifico}{State of the art}}
Quando nel 1998 l'esperimento Super-Kamiokande ha per la prima volta osservato le oscillazioni del neutrino\cite{fukuda:prl98}, si è dimostrato inequivocabilmente che tale particella possiede una massa. 
%Tuttavia, dal momento che tale fenomeno è funzione della differenza dei quadrati delle masse, sono necessarie altre tipologie di esperimenti per accedere alla scala di massa assoluta.
Tuttavia, essendo la probabilità di oscillazione funzione della differenza dei quadrati delle masse, tale fenomeno non permette di conoscere la scala di massa assoluta. 
Altre tipologie di esperimenti sono, dunque, necessarie per accedere al valore della massa del neutrino.
Fra i processi capaci di fornire questa informazione, particolare importanza assume il doppio decadimento beta senza l'emissione di neutrini (\doppiobeta), poiché la sua osservazione permetterebbe di chiarire in modo incontrovertibile se il neutrino è una particella di Dirac o di Majorana.
Inoltre, dal momento che tale processo viola la conservazione del numero leptonico, esso costituisce uno degli esempi più rilevanti di fisica oltre il Modello Standard (MS).
Per queste ragioni il \doppiobeta{} ha attratto a sè grande interesse da parte della comunità scientifica e nell'ultimo decennio innumerevoli esperimenti sono nati in tutto il mondo per osservarlo per la prima volta.
%come testimoniano gli innumerevoli esperimenti volti a misurarne il tempo di dimezzamento. 
%
%
%Il grande interesse della comunità scientifica sul \doppiobeta{} è testimoniato dagli innumerevoli esperimenti che in tutto il mondo provano a misurarne il tempo di dimezzamento.
% OPPURE POTREI SCRIVERE
%Questo processo ha attratto grande interesse da parte della comunità scientifica, come testimoniano gli innumerevoli esperimento che in tutto il mondo provano a misurarne il tempo di dimezzamento.
%
%
%
%Come evidenziato nella Sezione~\ref{sez:progetto_numen}, nell'espressione della probabilità di decadimento del \doppiobeta{} è presente un termine legato alla transizione del nucleo atomico dallo stato iniziale a quello finale. L'accesso per via sperimentale a tale termine è l'obiettivo principale del progetto NUMEN\cite{cappuzzello:epja18} (NUclear Matrix Elements for Neutrinoless double beta decay).  
%
%Come evidenziato nel seguito di questo capitolo, noto il tempo di dimezzamento del \doppiobeta{}, la deduzione della massa del neutrino è subordinata alla conoscenza dell'elemento di matrice che esprime la transizione del nucleo atomico dallo stato iniziale a quello finale. 
%
%L'accesso per via sperimentale a tale elemento di matrice è l'obiettivo principale del progetto NUMEN\cite{cappuzzello:epja18} (NUclear Matrix Elements for Neutrinoless double beta decay).
%
Dal momento che il \doppiobeta{} prevede transizioni fra nuclei atomici, una sua completa descrizione non può prescindere dalla struttura nucleare; in particolare, come mostrato nella~\ref{eq:rate_doppio_beta}, il tempo di dimezzamento dipende dall'elemento di matrice nucleare del processo.
%che esprime la transizione del nucleo dallo stato iniziale a quello finale. 
Il progetto NUMEN\cite{cappuzzello:epja18} (NUclear Matrix Elements for Neutrinoless double beta decay) ha come obiettivo principale l'accesso per via sperimentale a tale elemento di matrice.
%, come verrà spiegato nel seguito di questo capitolo.
\vspace{1cm}
 
In questo capitolo, dopo aver presentato le principali caratteristiche del \doppiobeta{}, vengono spiegate le motivazioni che hanno portato alla nascita di NUMEN, descrivendo le ambiziose sfide scientifiche e tecnologiche che il progetto intende affrontare e sottolineando l'importanza delle simulazioni all'interno di tale scenario.


%Il presente lavoro di tesi si colloca all'interno del progetto NUMEN\cite{cappuzzello:epja18} (NUclear Matrix Elements for Neutrinoless double beta decay), il quale propone un nuovo metodo per estrarre informazioni basate sui dati sperimentali sugli elementi di matrice nucleare che entrano in gioco nell'espressione del rate di dimezzamento del doppio decadimento beta senza neutrini (\doppiobeta). 
%Il \doppiobeta{} costituisce una delle aree di interesse più importanti della fisica contemporanea, come testimoniano gli innumerevoli esperimenti che nel mondo mirano alla sua scoperta.
%Come testimoniano gli innumerevoli esperimenti che nel mondo mirano alla sua scoperta, il \doppiobeta{} costituisce una delle aree di interesse più importanti della fisica contemporanea, dal momento che diverse questioni aperte del Modello Standard potrebbero trovare risposta nel caso in cui venisse osservato.


\section{\iflanguage{italian}{Il doppio decadimento beta senza neutrini}{The neutrinoless double beta decay}} \label{sez:doppio_beta_senza_neutrini}

L'idea del doppio decadimento beta fu per la prima volta suggerita da Maria Goeppert-Mayer nel 1935 in un articolo in cui si calcolava la probabilità di emissione simultanea di due elettroni e due \textcolor{red}{anti-neutrini (nell'articolo lei dice neutrini)} come un effetto del secondo ordine della teoria di Fermi del decadimento beta\cite{goeppert-mayer:pr35}. 
%Tale processo, oggi noto come doppio decadimento beta con due neutrini ($ 2\nu\beta\beta $), è contemplato all'interno del MS come un effetto del secondo ordine del decadimento beta.
Tale processo, oggi noto come doppio decadimento beta con due neutrini ($ 2\nu\beta\beta $), è contemplato all'interno del MS ed è stato osservato in undici isotopi, diventando il più raro e lento fenomeno naturale conosciuto.
 
%Il doppio decadimento beta con due neutrini ($ 2\nu\beta\beta $) è previsto all'interno del MS come un effetto del secondo ordine


%Il \doppiobeta{} è invece un processo proibito dal MS ed è possibile soltanto se il neutrino possiede una massa e coincide con la propria antiparticella, ovvero ....
Il \doppiobeta{} fu per la prima volta proposto da Furry nel 1939\cite{furry:pr39}, a seguito di un articolo di Majorana del 1937\cite{majorana:nc37} in cui il fisico catanese formulava l'ipotesi che il neutrino coincidesse con la propria antiparticella, ovvero fosse una \emph{particella di Majorana}. 
%soltanto se il neutrino possiede una massa ed è una particella di Majorana il \doppiobeta{} può avvenire.
%Nell'articolo di Furry veniva evidenziato come il \doppiobeta{} avesse un ruolo cruciale per fare luce sulla natura del neutrino; tale fenomeno è, infatti, possibile soltanto se il neutrino possiede una massa ed è una particella di Majorana. 
Nell'articolo di Furry veniva evidenziato il ruolo cruciale del \doppiobeta{} nella chiarificazione della natura del neutrino; il fenomeno in questione è, infatti, possibile soltanto se il neutrino possiede una massa ed è una particella di Majorana. 



%Proposto per la prima volta da Furry nel 1939\cite{furry:pr39}, il \doppiobeta{} è un processo di decadimento che può avvenire in uno dei modi seguenti:
%\begin{IEEEeqnarray}{rll}
%	& (A, Z) \rightarrow (A, Z+2) + 2e^{-}  & \\
%	& (A, Z) \rightarrow (A, Z-2) + 2e^{+}  & 
%\end{IEEEeqnarray}
%In letteratura il primo tipo di decadimento viene solitamente indicato con $\beta^-\beta^-$, mentre il secondo con $\beta^+\beta^+$.
%Proposto per la prima volta da Furry nel 1939\cite{furry:pr39}, il \doppiobeta{} è un processo di decadimento in cui due neutroni (protoni) in un nucleo atomico si trasformano in due protoni (neutroni) emettendo due elettroni (positroni) e nessun anti-neutrino (neutrino).
%Esso è possibile soltanto se il neutrino ha massa e coincide con la propria antiparticella, ovvero se è una particella di Majorana.
Il \doppiobeta{} è un processo di decadimento in cui due neutroni (protoni) in un nucleo atomico si trasformano in due protoni (neutroni) emettendo due elettroni (positroni) e nessun anti-neutrino (neutrino).
%Dal momento che vengono prodotti due elettroni, la conservazione del numero leptonico viene violata di due unità, rendendo il processo proibito secondo il~MS.
La creazione di due leptoni senza la presenza della corrispondente componente antileptonica implica che la conservazione del numero leptonico venga violata di due unità, rendendo il processo proibito secondo il MS. 
Sebbene fino ad oggi tale violazione non sia mai stata osservata, le teorie che descrivono l'unificazione dell'interazione elettrodebole e quella forte (Grand Unification Theories, GUTs) sono concordi nell'affermare che, ad energie dell'ordine di $10^{15}$ GeV, il numero leptonico cessa di essere un buon numero quantico\cite{pirro:epja06}. 
Ciò significa che il \doppiobeta{} potrebbe aprire la via verso una GUT delle interazioni fondamentali e svelare l'origine dell'asimmetria materia-antimateria presente nell'Universo\cite{vergados:ijmpe16}.
%\vspace{1cm}

Il rate di dimezzamento $ \left[ T_{1/2} \right]^{-1} $ del processo può essere espresso come il prodotto di tre fattori, ovvero
\begin{equation} \label{eq:rate_doppio_beta}
	\left[ T_{1/2} \right]^{-1} \; = \; G^{0 \nu} \: \left| M^{0 \nu} \right|^2 \: \left| f ( m_i, U_{ei}, \xi_i ) \right|^2 
\end{equation}
laddove $G^{0 \nu}$ è il fattore cinematico di spazio delle fasi dei due elettroni emessi; $ f ( m_i, U_{ei}, \xi_i ) $ è un termine contenente una combinazione delle masse $m_i$ delle tre specie di neutrini, dei coefficienti di mixing $U_{ei}$ della matrice PMNS e delle fasi di Majorana $\xi_i$; $M^{0 \nu}$ rappresenta l'ampiezza di probabilità di transizione del nucleo dallo stato iniziale $\phi_i$ a quello finale $\phi_f$, ossia
\begin{equation}
	M^{0 \nu} = \bra{\phi_f} \hat{O}^{0 \nu \beta \beta} \ket{\phi_f} 
\end{equation}
in cui $\hat{O}^{0 \nu \beta \beta}$ è l'operatore che descrive il \doppiobeta{}. 
Ad oggi i numerosi esperimenti che tentano di misurare il tempo di dimezzamento del processo sono stati in grado di fornire soltanto dei limiti inferiori; i più recenti risultati affermano che, al 90\% di livello di confidenza, $T_{1/2}$ deve essere maggiore di $8.0 \cdot 10^{25}$~yr nel caso del \ce{^{76}Ge}\cite{agostini:prl18}, e di $1.1 \cdot 10^{26}$~yr nel caso del \ce{^{136}Xe}\cite{gando:prl16}. Tali valori corrispondono ad un limite superiore per la massa del neutrino compreso tra 120 -- 260~meV nel primo caso e tra 50 -- 160~meV nel secondo.




La quantità $M^{0 \nu}$, nota in letteratura come \emph{elemento di matrice nucleare} (\emph{Nuclear Matrix Element}, NME), viene attualmente valutata attraverso avanzati metodi di calcolo, come ad esempio la Quasi-particle Random Phase Approximation (QRPA), il Large-scale Shell Model, l'Interacting Boson Model (IBM), l'Energy Density Functional (EDF) e i calcoli Ab-initio (\textcolor{red}{aggiungere ref??}). I vari metodi differiscono essenzialmente per il model space adottato, proponendo schemi di troncamento diversi a seconda dei gradi di libertà considerati rilevanti. 
Sebbene accurate informazioni provenienti da esperimenti di singolo scambio di carica (Single Charge Exchange, SCE), reazioni di transfer e cattura elettronica siano state utilizzate per porre dei vincoli ai calcoli teorici, le differenze tra i modelli sono ancora piuttosto grandi, tanto da osservare in alcuni casi discrepanze di un fattore due o tre, come si può evincere dalla Figura~\ref{fig:NME}. 

\begin{figure} [!t]
	\centering
	\includegraphics[scale=0.4]{Grafici/NME.png}
	\caption{I valori dei NMEs in funzione del numero di neutroni calcolati secondo i modelli IBM-2 \cite{barea:prc13}, QRPA-T\"{u}\cite{simkovic:prc13} e ISM\cite{menendez:npa08}. Figura tratta da~\cite{barea:prc15}.} \label{fig:NME}
\end{figure}



\section{\iflanguage{italian}{Reazioni di DCE e \doppiobeta}{DCE reactions and \doppiobeta}}


%In questo scenario appare evidente la necessità di dedurre dai dati sperimentali nuove informazioni, così da imporre limiti più stringenti ai modelli.  
Da quanto appena detto appare evidente la necessità di imporre limiti più stringenti ai modelli teorici, deducendo dai dati sperimentali nuove informazioni. 
%infatti, nonostante i NMEs non siano direttamente misurabili, sotto opportune condizioni e grazie a modelli teorici appropriati è possibile desumerne il valore tramite misure sperimentali di sezioni d'urto assolute.
In questa prospettiva, le reazioni di \emph{doppio scambio di carica} (Double Charge Exchange, DCE), ovvero le reazioni in cui la carica nucleare cambia di due unità lasciando invariato il numero di massa, si configurano come un potente strumento d'indagine sul \doppiobeta; 
%A causa della bassa sezione d'urto di tali processi, al fine di identificare le reazioni di DCE è essenziale la misura gli spettri energetici con grande risoluzione e le sezioni d'urto assolute ad angoli prossimi a zero.
infatti, sebbene i due processi siano mediati da interazioni differenti, ci sono diverse importanti similarità fra loro.
In primo luogo, gli stati nucleari inziali e finali del DCE coincidono con quelli del \doppiobeta{}, in quanto in entrambi i casi avviene la trasformazione di due neutroni (protoni) in due protoni (netroni). 
Un'altra significativa somiglianza riguarda gli operatori di transizione, i quali in tutte e due i processi contengono le componenti a corto range di Fermi, Gamow-Teller e tensoriale di rango-2, con un peso relativo che nelle reazioni di DCE dipende dall'energia incidente. 
Inoltre, in entrambi i casi nel canale intermedio virtuale l'impulso lineare è molto grande, dell'ordine di 100~MeV/c\cite{barea:prl12}. Questo è un aspetto cruciale, poiché significa che sia le reazioni di DCE sia il \doppiobeta{} sondano stati ad alto impulso della funzione d'onda nucleare, mentre altri meccanismi non ne sono in grado\cite{puppe:prc11}.

Le reazioni di DCE possono essere uno strumento utile per la comprensione del \doppiobeta{} in quanto permettono di studiare un fenomeno estremamente raro attraverso un meccanismo che, essendo guidato dall'interazione forte, possiede dei tempi caratteristici molto più brevi. 
In aggiunta, il processo di DCE ha il vantaggio di poter essere studiato attraverso misure sperimentali in un laboratorio, in una condizione che consente di tenere sotto controllo alcuni dei parametri fondamentali.
Tuttavia, l'analisi delle reazioni di DCE presenta anche degli inconvenienti: in primis, tali reazioni sono caratterizzate da sezioni d'urto molto basse, tipicamente di alcune decine di nb.
Di conseguenza, per accumulare una statistica sufficiente possono essere necessari lunghi tempi di raccolta dei dati e, a seconda dell'isotopo studiato, fasci di intensità molto grande.
Inoltre, al fine di identificare le reazioni di interesse è essenziale misurare con grande risoluzione e accuratezza sia gli spettri energetici sia le sezioni d'urto assolute ad angoli prossimi a zero. 
%Inoltre, risulta necessario misurare anche gli altri canali di reazione, in modo da  identificare e quantificare i processi di transfer di nucleoni multi-step che concorrono al meccanismo diretto. Questi contributi possono essere minimizzati grazie ad una scelta opportuna del sistema proiettile-target e dell'energia incidente.
Infine, non bisogna dimenticare che eventi così rari sono sommersi da un grande background; risulta, dunque, necessario misurare anche gli altri canali di reazione, in modo da poter identificare e quantificare i processi di transfer di nucleoni multi-step che concorrono al meccanismo diretto.




\section{\iflanguage{italian}{Il progetto NUMEN}{The NUMEN project}} \label{sez:progetto_numen}

Il progetto NUMEN propone un nuovo metodo per estrarre informazioni basate sui dati sperimentali (\textcolor{red}{meglio scrivere data-driven?}) sui NMEs che entrano in gioco nel calcolo del rate di dimezzamento del \doppiobeta{}. 
%utilizzando misure accurate di sezioni d'urto di reazioni di DCE indotte da ioni pesanti. 
Per raggiungere tale scopo si intende misurare con grande accuratezza le sezioni d'urto di reazioni di DCE indotte da ioni pesanti, esplorando a diverse energie del fascio incidente \emph{tutti} gli isotopi coinvolti negli esperimenti presenti e futuri sul \doppiobeta{}.
%In particolare, è importante verificare se le sezioni d'urto misurate del DCE sono legate ai NMEs del \doppiobeta{} come una funzione lentamente variabile dell'energia del proiettile e della massa del sistema.%cioè tipo $M^{DCE} \propto f(E_p, A, M^{0 \nu})$
%In tal caso, sarebbe possibile accedere agli elementi di matrice del \doppiobeta{} tramite misure di sezioni d'urto sperimentali. Dal punto di vista teorico, è necessario descrivere accuratamente il meccanismo di reazione, che deve essere fattorizzato in una parte di reazione ed una di struttura nucleare, con quest'ultima a sua volta fattorizzata nel termine del proiettile e in quello del bersaglio.

Il principale, e più ambizioso, obiettivo di NUMEN è l'accesso ai NMEs del \doppiobeta{} attraverso un approccio sperimentale. A tal fine bisogna verificare se gli elementi di matrice del DCE sono legati ai NMEs del \doppiobeta{} come una funzione lentamente variabile dell'energia del proiettile e della massa del sistema.
Qualora questa ipotesi fosse verificata, allora sarebbe possibile dedurre i NMEs del \doppiobeta{} a partire da misure di sezioni d'urto.
% ******* Prima avevo scritto questo:
%Se i risultati sperimentali confermassero che gli elementi di matrice del DCE sono legate ai NMEs del \doppiobeta{} come una funzione lentamente variabile dell'energia del proiettile e della massa del sistema, allora sarebbe possibile dedurre questi ultimi a partire da misure di sezioni d'urto. 
Ciò richiede che il meccanismo di reazione possa essere descritto come il prodotto di un fattore dovuto alla mera reazione e di uno relativo alla struttura nucleare, con quest'ultimo a sua volta fattorizzato in un termine del proiettile e in uno del bersaglio.
%Tale approccio si è dimostrato valido nel caso delle reazioni di singolo scambio di carica (vv. articolo Taddeucci 1987). 
Dunque, lo sviluppo di una teoria microscopica coerente della reazione di DCE è parte indispensabile del progetto. 
Dal punto di vista sperimentale, la verifica della validità di questa ipotesi richiede la costruzione di una sistematica di dati, che comprenda tutti gli isotopi soggetti al \doppiobeta.
%Sebbene per alcuni casi l'attuale apparato sperimentale possa essere sufficiente, tale studio sistematico richiede l'utilizzo di fasci di intensità molto più elevate di quelle al momento disponibili.
Poiché, come accennato nella sezione precedente, la maggior parte dei processi di DCE presenta sezioni d'urto estremamente basse, tale studio sistematico richiede l'utilizzo di fasci di intensità molto più elevate di quelle al momento disponibili.
In quest'ottica rientra la grande opera di ristrutturazione delle due principali componenti sperimentali: il Ciclotrone Superconduttore (CS) K800 e lo spettrometro magnetico MAGNEX.
%, affrontando le sfide connesse alla ricerca di fenomeni tanto rari, come la bassa sezione d'urto, la grande quantità di background, la necessità di alta risoluzione e sensibilità. 

Altro importante obiettivo di NUMEN consiste nella validazione delle teorie di struttura nucleare che si occupano di calcolare i NMEs del \doppiobeta{};
% ****** Prima avevo scritto:
%; infatti, poiché gli elementi di matrice del DCE e quelli del \doppiobeta{} contengono le stesse funzioni d'onda iniziali e finali e operatori di transizione con struttura simile, la misura di sezioni d'urto assolute può sondare la bontà dei model space adottati dai diversi metodi di calcolo.
infatti, gli elementi di matrice del DCE e quelli del \doppiobeta{} contengono le stesse funzioni d'onda iniziali e finali e operatori di transizione con struttura simile. Se scegliendo un determinato modello di struttura nucleare (con i relativi troncamenti alla funzione d'onda many-body) si trova un buon accordo con i dati sperimentali sulla sezione d'urto del DCE, allora quello stesso model space deve descrivere bene le funzioni d'onda del \doppiobeta.
% Dalla tesi di Ale: "Validare con i dati sperimentali l’applicazione di certi tagli sullo spazio di modello usato nell’analisi dei dati di DCE serve a validare la scelta dello stesso spazio di modello quando l’operatore non è più quello del doppio scambio di carica ma quello del decadimento 0νββ. In questo senso risulta essenziale avere il pieno controllo sulla componente di reazione della sezione d’urto."
Quindi, una volta scelte queste ultime dal confronto con le sezioni d'urto del DCE, le stesse possono essere impiegate per i NMEs del \doppiobeta{}. 

%Infine, NUMEN potrebbe fornire informazioni sulla sensibilità necessaria per la misura del tempo di dimezzamento del \doppiobeta{} a seconda dell'isotopo utilizzato. 
%Infine, NUMEN potrebbe fornire informazioni importanti sui diversi isotopi utilizzati nella ricerca del \doppiobeta{}, perché, facendo il rapporto delle sezioni d'urto assolute misurate negli esperimenti di DCE, si ottiene una stima di quanto il processo sia probabile indipendentemente dal modello adottato. Questa procedura, che consente di ridurre la presenza di eventuali errori sistematici poiché nel rapporto i due contributi si compensano, potrebbe 
Infine, NUMEN potrebbe fornire informazioni importanti sui diversi isotopi utilizzati nella ricerca del \doppiobeta{}, perché il rapporto delle sezioni d'urto assolute misurate negli esperimenti di DCE offre una stima di quanto il processo sia probabile indipendentemente dal modello assunto. 
%Questa procedura, che consente di ridurre la presenza di eventuali errori sistematici poiché nel rapporto i due contributi si compensano, potrebbe permettere di confrontare
Questa procedura consente di ridurre la presenza di eventuali errori sistematici poiché nel rapporto i due contributi si compensano.
Tale tipologia di analisi potrebbe avere un grande impatto sui futuri esperimenti sul \doppiobeta{}, in quanto potrebbe dare indicazioni su quale isotopo può essere il miglior candidato alla scoperta del processo e sulla sensibilità necessaria per la sua osservazione. 


Gli ambiziosi obiettivi di NUMEN pongono davanti numerose sfide, che richiedono lo sviluppo e l'utilizzo di tecniche innovative sia nel campo teorico sia in quello sperimentale. 
%In particolare, dal momento che il progetto prevede lo studio di tutti gli isotopi candidati al \doppiobeta{}, è necessario utilizzare fasci di intensità molto più alta di quella attualmente disponibile. 
%In questo contesto si inquadra il previsto upgrade delle infrastrutture dei Laboratori Nazionali del Sud (LNS).
%In particolare, dal momento che per studiare tutti gli isotopi candidati al \doppiobeta{} sono necessari fasci ad alta intensità, è fondamentale lo sviluppo di rivelatori capaci di sostenere un alto rate di conteggi.
%In particolare, dal momento che verranno utilizzati fasci ad alta intensità, per il progetto è fondamentale lo sviluppo di rivelatori capaci di sostenere un alto rate di conteggi; infatti, 
%In particolare, la necessità di utilizzare fasci ad elevata intensità rende di fondamentale importanza lo sviluppo di tecnologie di frontiera nell'ambito dei rivelatori ad alto rate di conteggi.
Fra queste, a causa dell'esigenza di utilizzare fasci ad elevata intensità, sono di fondamentale importanza la ricerca e lo sviluppo (R\&D) di tecnologie di frontiera nell'ambito dei rivelatori ad alto rate di conteggi.
%In questo tipo di attività le simulazioni costituiscono un potente strumento per valutare se le performance di un sistema di rivelazione possono soddisfare ai requisiti necessari, evitando così di dedicare tempo e risorse su soluzioni inefficaci.
In questo tipo di attività le simulazioni costituiscono un potente strumento per valutare se una soluzione può soddisfare ai requisiti necessari, evitando così di dedicare tempo e risorse su proposte inefficaci.

In questo contesto si colloca il ruolo del presente lavoro di tesi, che ha contribuito all'analisi attraverso simulazioni Monte Carlo delle prestazioni di un sistema di rivelazione a stato solido per l'identificazione di ioni pesanti.
%All'interno del progetto NUMEN, il presente lavoro di tesi ha contribuito all'analisi attraverso simulazioni Monte Carlo delle prestazioni di un sistema di rivelazione a stato solido per l'identificazione di ioni pesanti.


%L'attuale sistema di rivelazione, basato su pad di silicio, verrebbe danneggiato da tali intensità, quindi servono rivelatori con robustezza alle radiazioni.
%In particolare, è previsto una profonda trasformazione delle due principali infrastrutture sperimentali dell'intero progetto: il Ciclotrone Superconduttore (CS) K800 e lo spettrometro magnetico MAGNEX. 
%Sebbene per alcuni casi 

%L'upgrade dei lns è parte integrante del progetto.

\section{\iflanguage{italian}{L'upgrade dell'apparato sperimentale}{Upgrade of the experimental set-up}}

Come anticipato nella sezione precedente, al fine di studiare in modo sistematico tutti gli isotopi candidati al \doppiobeta{} è necessario utilizzare fasci di intensità molto più alte di quelle disponibili con l'attuale infrastruttura. Dunque, parte integrante di NUMEN è l'upgrade delle due componenti chiave del progetto, il CS e MAGNEX. 
È previsto che, alla fine del processo di ristrutturazione, l'apparato sperimentale possa essere in grado di lavorare con una corrente aumentata di due o tre ordini di grandezza, passando dalle attuali $10^{12}$~pps a circa $10^{14}$~pps (\textcolor{red}{numeri giusti?}).
%Poiché l'aumento della corrente deve essere di due o tre ordini di grandezza
Questo obiettivo può essere raggiunto soltanto a seguito di importanti cambiamenti nelle tecnologie utilizzate nell'estrazione e nel trasporto del fascio, nella realizzazione dei bersagli e nel sistema di rivelazione degli eiettili (\textcolor{red}{eiettili non esiste in italiano: lo uso lo stesso?}). 
In particolare, per quanto riguarda quest'ultimo aspetto, i principali cambiamenti previsti sono:
\begin{itemize}
	\item[--] l'aumento della massima rigidità magnetica accettata;
	\item[--] la sostituzione dell'attuale tracciatore a gas, basato su una tecnologia a fili, con un sistema che utilizza i rivelatori Thick-GEM (\textcolor{red}{giusto?});
	\item[--] la sostituzione dell'attuale muro di rivelatori a pad di silicio con una matrice di rivelatori di più piccola taglia e con migliori proprietà di resistenza alle radiazioni;
	\item[--] lo sviluppo di una matrice di rivelatori attorno al bersaglio per la misurazione dei raggi gamma emessi nella diseccitazione degli stati nucleari popolati nelle reazioni di DCE.
\end{itemize}
%Dunque, la necessità di sostenere alti rate di particelle porterà ad un profondo cambiamento dell'attuale rivelatore di piano focale (Focal Plane Detector, FPD) di MAGNEX, descritto nel capitolo successivo (\textcolor{red}{aggiungere la sezione}).
Dunque, al fine di raggiungere gli obiettivi preposti da NUMEN è necessaria una profonda trasformazione dell'attuale rivelatore di piano focale (Focal Plane Detector, FPD) di MAGNEX, descritto nel capitolo successivo (\textcolor{red}{agg sezione}).


%Dei cambiamenti precedentemente elencati particolare attenzione merita quello del tracciatore a gas 
%È importante sottolineare un aspetto dei cambiamenti precedentemente elencati: la sostituzione dell'attuale tracciatore a gas con un sistema 
Dei cambiamenti precedentemente elencati è importante sottolineare un aspetto: il presente sistema di tracciamento, oltre a fornire accurate informazioni sulla posizione, è sensibile alla perdita di energia degli ioni nel gas. Esso viene, quindi, utilizzato anche come stadio $\Delta E$ per l'identificazione in numero atomico ($Z$) dei prodotti di reazione.
La tecnologia delle Thick-GEM, scelta perché promette buone proprietà di misura della posizione anche in presenza di alti rate, non è invece in grado di dare informazioni sull'energia persa dagli ioni.
Inoltre, gli attuali rivelatori a larga area al silicio, usati per misurare l'energia residua ($ E_r $), non soltanto verrebbero danneggiati da rate così alti, ma sarebbero anche soggetti ad un significativo pile-up a causa delle loro grandi dimensioni.
Di conseguenza, per non diminuire le attuali capacità complessive di identificazione delle particelle (Particle IDentification, PID) è necessario introdurre nel FPD un sistema dedicato a questo scopo.



\subsection{\iflanguage{italian}{Il nuovo sistema di identificazione delle particelle}{The new system of particle identification}}


Il requisito fondamentale che il nuovo sistema di PID deve soddisfare è l'identificazione degli ioni nella regione dell'ossigeno (O), del fluoro (F) e del neon (Ne). 
Oltre a questo, esso deve possedere le seguenti caratteristiche:
\begin{enumerate}
	\item alta resistenza alle radiazioni (\textcolor{red}{aggiungere numeri per quantificare?});
	\item la risoluzione energetica deve essere sufficientemente buona da garantire una chiara identificazione dei prodotti di reazioni di interesse per NUMEN;
	\item il grado di segmentazione deve essere scelto in modo da mantenere la probabilità di eventi con double-hit inferiore al 3\%;
	\item l'efficienza geometrica deve essere sufficientemente alta da ridurre al minimo il background costituito dagli eventi con raccolta di carica parziale;
	\item lo spessore dei rivelatori deve riuscire a fermare gli eiettili di interesse in un grande range di energia di incidenza;
	\item i rivelatori devono essere facilmente costruibili e maneggiabili e avere un costo ragionevole.
\end{enumerate}


Dopo aver valutato diverse opzioni, tra cui i phoswich, si è scelto di utilizzare un muro di telescopi $ \Delta E - E $ a stato solido.
Negli esperimenti di fisica nucleare questo tipo di sistema è tipicamente composto da uno stadio $\Delta E$ sottile al silicio, seguito da un rivelatore spesso al silicio o da uno scintillatore per fermare lo ione. (\textcolor{red}{C'è bisogno di mettere un accenno alla Bethe-Bloch?}) 
Poiché, come già detto in precedenza, i rivelatori al silicio non possiedono le proprietà di resistenza alle radiazioni necessarie per il progetto, la scelta è oggi indirizzata verso un telescopio in cui il primo stadio è costituito da un rivelatore sottile (100~$\mu $m) al carburo di silicio\cite{tudisco:sensors18} (SiC), mentre il rivelatore di stop è uno scintillatore allo ioduro di cesio (CsI) spesso 1~cm.

%Il principale scopo del presente lavoro di tesi è stato la valutazione 
Al fine di verificare se questa scelta possa garantire le performance di PID e la risoluzione energetica necessarie per gli obiettivi del progetto, è stata implementata una simulazione Monte Carlo sulla piattaforma GEANT4\cite{agostinelli:nima02}.
Tale simulazione, che costituisce l'argomento centrale del presente lavoro di tesi, ha anche lo scopo di valutare la migliore soluzione in termini di granularità, stimando il numero di eventi con raccolta di carica incompleta.








\section{\iflanguage{italian}{Le fasi del progetto}{The phases of the project}}

Il progetto NUMEN è articolato in quattro fasi, di cui verranno esposti i tratti più importanti. 

La \emph{Fase 1}, già completata, ha dimostrato, grazie all'esperimento pilota \ce{^{40}Ca}(\ce{^{18}O},\ce{^{18}Ne})\ce{^{40}Ar}, che è possibile estrarre informazioni sulle funzioni d'onda nucleari del \doppiobeta{} tramite lo studio di reazioni di DCE.

La \emph{Fase 2}, attualmente in corso, prevede lo svolgimento di una campagna sperimentale su alcuni isotopi di interesse, scelti come compromesso tra la rilevanza di tali isotopi per gli esperimenti sul \doppiobeta{} e le esigenze tecniche. I primi sistemi oggetto di studio sono stati $^{116}\mbox{Cd}\,  - \, ^{116}\mbox{Sn} $ e $^{76}\mbox{Ge}\,  - \, ^{76}\mbox{Se} $, sondati attraverso le reazioni (\ce{^{20}Ne}, \ce{^{20}O}) e (\ce{^{18}O}, \ce{^{18}Ne}) per esplorare il meccanismo di DCE in entrambe le direzioni. Prossimamente verrà effettuato un esperimento sulla reazione \ce{^{48}Ti}(\ce{^{18}O},\ce{^{18}Ne})\ce{^{48}Ca}. 
%Durante questa fase verrà anche ottimizzata la strategia di analisi dei dati.
Della Fase~2 fa parte anche l'attività di R\&D su rivelatori, materiali e tecnologie precedentemente descritta.

La \emph{Fase 3} comprende sia lo smontaggio dell'attuale apparato sperimentale sia l'assemblaggio del nuovo. In questa fase avrà luogo anche l'upgrade del CS e della linea di trasporto. La durata prevista è di 18 - 24 mesi.
%La \emph{Fase 3} è dedicata all'upgrade del CS e di MAGNEX: in questa fase 


La \emph{Fase 4} prevede una serie di campagne sperimentali che, grazie alle acquisite condizioni di alta intensità del fascio, comprenderà tutti gli isotopi di interesse per il \doppiobeta{}. 
Questa fase sarà dedicata al calcolo della sezione d'urto assoluta di DCE. Se l'analisi teorica sarà riuscita a sviluppare una descrizione microscopica delle reazioni di DCE, allora sarà possibile avere accesso ai NMEs del \doppiobeta{}, principale obiettivo di NUMEN.










\clearpage


\chapter{\iflanguage{italian}{L'apparato sperimentale}{Experimental set-up}}


%L'apparato sperimentale attualmente in uso ai LNS-INFN nell'ambito della Fase~2 del progetto NUMEN, costituito principalmente dallo spettrometro MAGNEX e dal Ciclotrone Superconduttore K800, viene brevemente illustrato nella prima parte di questo capitolo.
L'apparato sperimentale attualmente in uso ai LNS-INFN nell'ambito della Fase~2 del progetto NUMEN è costituito principalmente dallo spettrometro MAGNEX e dal CS.
Poiché la descrizione di tale apparato non costituisce l'argomento primario del presente lavoro di tesi, nella prima parte del capitolo ne vengono discusse soltanto le caratteristiche principali, rimandando alla vasta letteratura sul tema per informazioni più dettagliate (ad esempio~\cite{cavallaro:epja12, carbone:epja12, cappuzzello:epja16, cunsolo:epjst07}).

%Nella prima parte di questo capitolo vengono illustrate le principali caratteristiche dell'apparato sperimentale attualmente in uso ai LNS-INFN nell'ambito della Fase~2 del progetto NUMEN.
%Nella seconda parte viene descritta la configurazione dell'apparato adottata in occasione del test sui telescopi SiC-CsI svolto ad Aprile 2018, sottolineando le differenze rispetto a quella consueta.
(\textcolor{red}{Sistemare questa parte})Nella seconda parte \textcolor{red}{si parla} del test sui telescopi SiC-CsI svolto ad Aprile~2018, esponendone le motivazioni, descrivendo la configurazione dell'apparato adottata e sottolineandone le differenze rispetto a quella consueta.


\section{\iflanguage{italian}{Lo spettrometro magnetico MAGNEX}{MAGNEX magnetic spectrometer}}

Lo spettrometro magnetico MAGNEX è un dispositivo ottico a grande accettanza costituito da un quadrupolo per la focalizzazione sull'asse verticale, seguito da un dipolo per la dispersione sul piano orizzontale.
Grazie alle sue peculiarità,  MAGNEX riesce ad offrire, in un angolo solido molto grande e in un ampio range energetico, un'ottima risoluzione in energia, angolo e massa.
Ciò lo rende uno strumento ideale per l'analisi di eventi caratterizzati da sezioni d'urto molto basse, come è già stato dimostrato in~\cite{cappuzzello:epja16,pereira:plb12,oliveira:jpg13}.
Inoltre, esso consente di effettuare misure fino a zero gradi, comprendendo, dunque, la regione angolare di massimo interesse per lo studio del DCE.

La caratteristica che rende MAGNEX uno strumento unico è l'implementazione di una innovativa tecnica di ricostruzione delle traiettorie degli ioni, che consente di correggere le inevitabili aberrazioni originate dalla grande accettanza del dispositivo.
Dunque, a differenza di altri spettrometri magnetici, per MAGNEX è importante determinare non soltanto il punto di impatto sul piano focale ma anche la traiettoria completa. Ciò significa che è necessario misurare quattro parametri: una coppia, chiamata $(x_{foc}, y_{foc})$, individua il punto di impatto, l'altra, indicata con $(\theta_{foc}, \phi_{foc})$, si riferisce rispettivamente all'angolo orizzontale e a quello verticale.
%Nel paragrafo successivo verrà esplicato in che modo vengono misurati tali parametri.
Il modo in cui tali parametri vengono misurati sarà esplicato nel paragrafo successivo.

In Figura~\ref{fig:magnex} è mostrata una foto di MAGNEX, in cui è possibile notare, andando da sinistra verso destra, la camera di scattering, il quadrupolo, il dipolo e il~FPD (\textcolor{red}{Aggiungere le scritte sull'immagine}).

\begin{figure} [!t]
	\centering
	\includegraphics[width=\textwidth, keepaspectratio]{Grafici/magnex.jpg}
	\caption{Lo spettrometro magnetico MAGNEX.} \label{fig:magnex}
\end{figure}



%\clearpage 

\subsection{\iflanguage{italian}{Il rivelatore di piano focale}{The Focal Plane Detector}} \label{sez:fpd}

\begin{figure} [!p]
	\centering
	\includegraphics[width=\textwidth, keepaspectratio]{Grafici/fpd.png}
	\caption{Rappresentazione schematica del FPD: a) vista laterale; b) vista dall'alto. Figura tratta da~\cite{cappuzzello:epja18}.} \label{fig:fpd}
\end{figure}

L'attuale FPD di MAGNEX, la cui rappresentazione schematica è riportata in Figura~\ref{fig:fpd}, è un sistema di rivelazione ibrido, costituito da un tracciatore a gas a bassa pressione e da un muro di rivelatori al silicio.
Esso è posizionato a 1.91~m dall'uscita del dipolo e, al fine di minimizzare gli effetti dovuti alle aberrazioni cromatiche\cite{cunsolo:nima01}, è inclinato di 59.2\textdegree{} rispetto ad un piano perpendicolare all'asse ottico.
Una finestra di mylar spessa 1.5~$\mu$m è utilizzata per contenere il gas, solitamente costituito da N35 isobutano, segnando l'ingresso nel volume attivo.




%Il tracciatore a gas è formato da un sistema di sei fili al tungsteno placcati in oro, posti al di sotto di un anodo segmentato in pad. 
%Il tracciatore a gas lavora secondo il principio tipico delle camera a deriva.
%Il tracciatore a gas è essenzialmente una camera a deriva, in cui un sistema misto di fili e pad consente la misura dei quattro parametri necessari.
%Il tracciatore a gas, che consente la ricostruzione tridimensionale della traiettoria degli ioni, è formato da sei fili al tungsteno, indicati con DC\ped{\textit{i}}, e da un anodo segmentato in pad. 
%Al di sopra di ciascun filo si trova una fila di 224 pad
%Il tracciatore a gas è essenzialmente una camera a deriva, in cui un sistema costituito da sei fili (DC\ped{\textit{i}}) e da un anodo segmentato in pad consente la misura dei quattro parametri necessari per la ricostruzione tridimensionale della traccia.
%In particolare, al di sopra di ciascun filo è presente una fila di pad, le quali sono orientate parallelamente all'asse ottico.
Il tracciatore a gas, che consente la ricostruzione tridimensionale della traiettoria degli ioni, è formato da sei fili proporzionali (DC\ped{\textit{i}}) e da un anodo segmentato in sei file di pad, disposti in modo che sopra ogni filo ci sia una fila di pad.
I fili, sfruttando il principio di lavoro delle camere a deriva, danno una misura di sei posizioni verticali ($Y_i$), mentre le pad permettono di determinare sei posizioni orizzontali ($X_i$).
%Una griglia di Frisch è posta al di sotto dei fili, in quanto questi vengono anche utilizzati per misurare l'energia persa dagli ioni nel gas.
Dal momento che i fili vengono anche utilizzati per misurare l'energia persa dagli ioni nel gas, una griglia di Frisch è posta al di sotto di essi.




Il muro di rivelatori al silicio è formato da 60 pad, organizzate in 20 colonne da 3 rivelatori ciascuna. (\textcolor{red}{Aggiungo qualche dettaglio in più?})
%Ogni pad ha un'area attiva di $70 \times 50$~mm\ap{2} ed è spessa 500~$\mu$m, sufficienti per fermare i prodotti di reazioni nel range energetico di interesse. 
Essi vengono utilizzati per fermare gli ioni, misurandone l'energia residua e producendo il segnale di trigger per l'acquisizione. 

Quando una particella carica, attraversando la finestra di mylar, entra nel volume attivo, produce nel gas coppie elettrone-ione positivo, le quali, sotto l'effetto di un campo elettrico costante, migrano rispettivamente verso la griglia di Frisch e il catodo. 
%mentre gli elettroni diffondono verso la griglia di Frisch, con una velocità che per questi ultimi è di circa $3 - 5 $~cm/$\mu$s. 
%La presenza di un campo elettrico costante provoca 
%La velocità di deriva tipica degli elettroni è di $3 - 5 $~cm/$\mu$s
Dopo aver attraversato la griglia, gli elettroni giungono in prossimità dei fili DC\ped{\textit{i}}, dove, a causa dell'elevato campo elettrico, danno luogo alla moltiplicazione a valanga. 
%La carica prodotta, proporzionale all'energia persa dalla particella nel gas, induce sulle pad una distribuzione di carica, della quale si calcola il baricentro. Questa operazione avviene per le sei file di pad, in modo tale che ai sei baricentri corrispondono sei misure di posizioni orizzantali $X_i$.
Alle tensioni e pressioni utilizzate, la carica secondaria prodotta genera un segnale proporzionale all'energia persa dalla particella nel gas. 
Poiché ciò avviene per ciascuno dei sei fili, si hanno sei segnali di perdita di energia, indicati con~$\Delta E_i$.

La stessa valanga induce sulle pad una distribuzione di carica, di cui si calcola, tramite un apposito software, il centro di gravità. Anche in questo caso l'operazione si ripete per le sei file di pad, così che vengono estratte sei misure di posizioni orizzontali~$X_i$.
A questo punto, effettuando un fit lineare sulle sei posizioni~$X_i$, si ottengono $x_{foc}$ e $\theta_{foc}$ rispettivamente dall'intercetta e dal coefficiente angolare della retta.

Superata la regione del tracciatore, la particella carica arriva al muro dei rivelatori al silicio, dove si ferma producendo un segnale proporzionale alla sua energia residua~$E_{resid}$. 
Lo stesso segnale viene utilizzato per calcolare l'intervallo di tempo impiegato dagli elettroni primari prodotti nel gas per raggiungere i fili DC\ped{\textit{i}}. 
%Dal momento che tale intervallo è proporzionale allo spazio percorso, si ottengono così sei misure di posizioni verticali~$Y_i$, dalle quali, grazie ad un fit lineare, si ricavano $y_{foc}$ e $\phi_{foc}$ in maniera analoga a quanto visto per $x_{foc}$ e $\theta_{foc}$.
Dal momento che la velocità di deriva è costante, tale intervallo di tempo è direttamente proporzionale allo spazio percorso. Si ottengono, dunque, sei misure di posizioni verticali~$Y_i$, dalle quali, grazie ad un fit lineare, si ricavano $y_{foc}$ e $\phi_{foc}$ in maniera analoga a quanto visto per $x_{foc}$ e $\theta_{foc}$.

È bene ricordare che, nelle camere a deriva a fili, la maggior parte del segnale è originata dal moto degli ioni positivi e non degli elettroni.
Di conseguenza, a causa della minore velocità degli ioni, tale tipologia di rivelatori può tipicamente sostenere rate dell'ordine di pochi~kHz. 
Questo aspetto, che costituisce una delle principali limitazioni all'intensità del fascio tollerabile dall'attuale FPD, deve essere superato nell'ottica della Fase~4. 
Nasce da qui l'esigenza di sostituire l'attuale tracciatore con un sistema in grado di lavorare con un rate elevato di ioni pesanti. 


\section{\iflanguage{italian}{Il rivelatore di piano focale di NUMEN}{NUMEN Focal Plane Detector}}


%\section{\iflanguage{italian}{Il test sui telescopi SiC-CsI}{Experimental setting for the test}}
\section{\iflanguage{italian}{I telescopi SiC-CsI}{SiC-CsI telescope detectors}}

%Ad Aprile 2018 è stato svolto un test sui primi due prototipi di telescopi SiC-CsI, allo scopo di confrontarne la risposta con il sistema attualmente utilizzato.
%Ad Aprile 2018 è stato svolto un test sui primi due prototipi di telescopi SiC-CsI, allo scopo di valutarne la risposta, confrontandola con quella del sistema attualmente utilizzato.

Ad Aprile 2018 i primi due prototipi di rivelatori al SiC per il progetto NUMEN sono stati completati e resi disponibili dalla \textcolor{red}{ST-Microelectronics} (\textcolor{red}{giusto?}).
%Dopo aver realizzato 
%Per verificare se la risposta di un telescopio formato da uno stadio $\Delta E$ al SiC seguito da un cristallo di CsI poteva garantire prestazioni di identificazione degli ioni confrontabili con quelle accessibili con l'attuale apparato, è stato svolto un test per una durata complessiva di cinque giorni.
%Dal momento che tali rivelatori non sono ancora uno standard nel mondo della fisica ma rappresentano una tecnologia nuova, è stato svolto un test per verificare se le loro prestazioni possono soddisfare le esigenze di NUMEN.
Dal momento che tali rivelatori non sono ancora uno standard nel mondo della fisica ma rappresentano una tecnologia di frontiera, è stato condotto un test per analizzarne la risposta nelle condizioni sperimentali tipiche della Fase~2 di NUMEN.
%Come illustrato nel Paragrafo~\ref{sez:sistema_identif_part}, i rivelatori al SiC sono stati scelti nell'ambito del progetto per assolvere al ruolo di sistema di PID
%Sono stati, dunque, assemblati due telescopi SiC-CsI, di cui si è studiata la risposta in termini di PID.
Come illustrato nel Paragrafo~\ref{sez:sistema_identif_part}, i rivelatori al SiC sono stati scelti nell'ambito del progetto come stadio~$\Delta E$ di un telescopio SiC-CsI per l'identificazione in numero atomico dei prodotti di reazione.
%Come illustrato nel Paragrafo~\ref{sez:sistema_identif_part}, i rivelatori al SiC sono stati scelti nell'ambito del progetto per identificare i prodotti di reazione
Lo scopo principale del test era, dunque, valutare la capacità di PID di questo sistema nella regione di interesse per NUMEN, confrontandola con quella accessibile con l'attuale apparato.

%Il fascio utilizzato nel test era costituito da \ce{^{20}Ne}, mentre i bersagli erano \ce{^{197}Au} e \ce{^{12}C}
Il test è stato svolto utilizzando un fascio di~\ce{^{20}Ne} a 20~MeV/u, incidente su \ce{^{197}Au} o \ce{^{12}C}, laddove i due bersagli avevano funzioni differenti: il primo serviva per la localizzazione dello scattering elastico, il secondo per favorire la formazione dei prodotti di reazione di interesse per NUMEN, ovvero quella dell'O, del F e del Ne.






%\subsection{\iflanguage{italian}{I telescopi SiC-CsI}{SiC-CsI telescope detectors}}

\subsection{\iflanguage{italian}{I rivelatori al carburo di silicio}{Silicon carbide detectors}}


%I dispositivi al SiC costituiscono oggi una promettente realtà nel campo dei rivelatori per la fisica, 
Grazie alle sue interessanti proprietà, il SiC costituisce oggi una promettente realtà nel campo della realizzazione di rivelatori per la fisica, configurandosi come una potenziale alternativa al silicio nelle applicazioni che richiedono una grande resistenza alle radiazioni.
La larghezza della sua gap, quasi doppia rispetto a quella del silicio, se da un lato aumenta l'energia media per la produzione di una coppia elettrone-lacuna, dall'altro riduce il numero di portatori di carica generati per agitazione termica, mantenendo, dunque, un ottimo rapporto segnale-rumore. 
Recenti test\cite{tudisco:sensors18} hanno dimostrato che con un rivelatore da 100~$\mu$m di spessore è possibile ottenere una risoluzione energetica dello 0.8\%~FWHM per le particelle $\alpha$ dell'\ce{^{241}Am} a 5486~keV.


Come la maggior parte dei rivelatori a semiconduttore, i rivelatori al SiC sono costituiti da una giunzione p-n, polarizzata inversamente per aumentare l'estensione della regione svuotata e per migliorare l'efficienza di raccolta dei portatori di carica.
Quando una particella carica attraversa il rivelatore, perde energia generando coppie elettrone-lacuna, le quali, in presenza di un campo elettrico, si muovono verso gli elettrodi, producendo il segnale.



%Gli attuali limiti tecnologici alla produzione di rivelatori al SiC sono legati alla capacità di controllare le dimensioni dell'area attiva e lo spessore dei dispositivi, dal momento che essi vengono realizzati tramite crescita epitassiale su un substrato di SiC.
Gli attuali limiti tecnologici alla produzione di rivelatori al SiC sono legati alla capacità di fabbricare dispositivi con area attiva e spessori relativamente grandi, dal momento che essi vengono realizzati tramite crescita epitassiale su un substrato di SiC.
%Ciò comporta anche la presenza di una regione in cui la ... ZONA PARZIALMENTE VIVA .... NON SO BENE COME INTRODURLA
Inoltre, proprio la presenza di tale substrato può rappresentare un problema ai fini della rivelazione, in quanto costituisce uno strato morto in cui le particelle perdono energia senza che questa possa essere misurata o possono perfino fermarsi.
%Tali problematiche sono state tenute in considerazione nella simulazione svolta per questo lavoro di tesi, la quale ne ha valutato l'impatto sulle performance del previsto sistema di rivelazione. 
La simulazione svolta per questo lavoro di tesi ha tenuto in considerazione tali limiti e problematiche, valutandone l'impatto sulle performance del previsto sistema di rivelazione e suggerendo le migliori specifiche tecniche ai fini del progetto. 



\begin{figure} [!t]
	\centering
	\includegraphics[width=\textwidth, keepaspectratio]{Grafici/sic.jpg}
	\caption{I rivelatori al carburo di silicio (SiC) utilizzati nel test: a sinistra il \emph{SiC~B}, a destra il \emph{SiC~A}.} \label{fig:sic}
\end{figure}


%I due prototipi di rivelatori al SiC utilizzati nel test sono mostrati in Figura~\ref{fig:sic}; essi avevano caratteristiche differenti: uno, chiamato \emph{SiC~A}, aveva un'area attiva non segmentata di $10 \times 10$~mm\ap{2}, con uno spessore di 10~$\mu$m ed uno strato morto di 100~$\mu$m; l'altro, indicato con \emph{SiC~B}, aveva un'area attiva segmentata in quattro pad, ciascuna di $5 \times 5$~mm\ap{2}, con uno spessore di 100~$\mu$m ed un strato morto di 350~$\mu$m.

I due prototipi di rivelatori al SiC utilizzati nel test sono mostrati in Figura~\ref{fig:sic}: come è possibile notare, essi differivano innanzitutto per la segmentazione dell'area attiva, in quanto uno, chiamato \emph{SiC~A}, era costituito da un'unica pad, mentre l'altro, indicato con \emph{SiC~B}, era suddiviso in quattro regioni. 
%Inoltre, mentre il SiC~A aveva un'area attiva di $10 \times 10$~mm\ap{2}, il SiC~B presentava 
%I due telescopi SiC-CsI utilizzati nel test
Entrambi i rivelatori avevano un'area attiva complessiva di $10 \times 10$~mm\ap{2}, laddove ogni pad del SiC~B aveva un'estensione di $5 \times 5$~mm\ap{2}.
%Un'ulteriore differenza riguardava gli spessori dei due rivelatori: mentre il SiC~A era spesso 10~$\mu$m con 100~$\mu$m di strato morto, il SiC~B aveva uno spessore di 100~$\mu$m con 350~$\mu$m di .
Un'ulteriore differenza riguardava lo spessore della regione attiva dei due rivelatori: per il SiC~A era di 10~$\mu$m, mentre per il SiC~B misurava 100~$\mu$m.
Infine, entrami i rivelatori avevano uno strato morto che, mentre per il SiC~A era spesso 100~$\mu$m, per il SiC~B aveva uno spessore di~350~$\mu$m.

In occasione del test si è scelto di utilizzare soltanto due delle quattro pad del SiC~B, le quali sono state cortocircuitate tra loro in modo tale che il rivelatore avesse una superficie sensibile di $10 \times 5$~mm\ap{2}.




%I due prototipi di rivelatori al SiC utilizzati nel test presentavano caratteristiche differenti: in primo luogo avevano spessore diversi, poiché uno era spesso 10~$\mu$m, mentre l'altro 100~$\mu$m.



\subsection{\iflanguage{italian}{Gli scintillatori allo ioduro di cesio}{Caesium iodide scintillation detector}}

%\subsection{\iflanguage{italian}{I telescopi SiC-CsI}{SiC-CsI telescope detectors}}

I due rivelatori al SiC sono stati assemblati insieme ai due scintillatori allo CsI mostrati in Figura~\ref{fig:csi}. 
Questi avevano dimensioni differenti: uno (\emph{CsI~A}) era grande $1 \times 1$~cm\ap{2}, l'altro (\emph{CsI~B}) era suddiviso in quattro regioni da $1.5 \times 1.5$~cm\ap{2} ciascuna. 
%La luce di scintillazione veniva letta in entrambi i casi con un fotodiodo da $1 \times 1$~cm\ap{2}, che 
%Ognuno dei due scintillatori era accoppiato ad un fotodiodo $1 \times 1$~cm\ap{2}
A causa delle dimensioni dei rivelatori al SiC, soltanto una delle quattro aree sensibili del CsI~B è stata utilizzata.
Ciascuno dei due cristalli era accoppiato ad un fotodiodo $1 \times 1$~cm\ap{2} per la lettura della luce di scintillazione. 


\begin{figure} [!t]
	\centering
	\includegraphics[width=\textwidth, keepaspectratio]{Grafici/csi.jpg}
	\caption{I cristalli di ioduro di cesio (CsI) utilizzati nel test: a sinistra il \emph{CsI~B}, a destra il \emph{CsI~A}.} \label{fig:csi}
\end{figure}



\begin{figure} [!t]
	\centering
	\includegraphics[width=\textwidth, keepaspectratio]{Grafici/telescopi.jpg}
	\caption{I due telescopi SiC-CsI utilizzati nel test: a sinistra il \emph{Tel~A}, a destra il \emph{Tel~B}.} \label{fig:telescopi}
\end{figure}

\clearpage

\section{\iflanguage{italian}{La configurazione dell'apparato sperimentale nel test}{Experimetal apparatus setting}}



%\cleardoublepage
%\phantomsection
%\chapter*{\iflanguage{italian}{Conclusioni}{Conclusions}}
%\addcontentsline{toc}{chapter}{\iflanguage{italian}{Conclusioni}{Conclusions}}

%\lipsum[20]


\cleardoublepage
\phantomsection
\addcontentsline{toc}{chapter}{\iflanguage{italian}{Bibliografia}{Bibliography}}
\bibliographystyle{mprsty}
\bibliography{biblio}


%\cleardoublepage
%\phantomsection
%\chapter*{\iflanguage{italian}{Ringraziamenti}{Acknowledgements}}
%\addcontentsline{toc}{chapter}{\iflanguage{italian}{Ringraziamenti}{Acknowledgements}}

%\lipsum[20]

\end{document}




