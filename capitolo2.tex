

Nella prima parte di questo capitolo vengono descritte le principali caratteristiche dell'apparato sperimentale attualmente in uso ai LNS-INFN nell'ambito della Fase~2 del progetto NUMEN.
Nella seconda parte viene descritta la configurazione dell'apparato adottata in occasione del test svolto ad Aprile 2018.



\section{\iflanguage{italian}{Lo spettrometro magnetico MAGNEX}{MAGNEX magnetic spectrometer}}

Lo spettrometro magnetico MAGNEX è un dispositivo ottico a grande accettanza costituito da un quadrupolo per la focalizzazione sull'asse verticale, seguito da un dipolo per la dispersione sul piano orizzontale.
Grazie alle sue caratteristiche uniche, descritte in dettaglio in \cite{cappuzzello:epja16,cunsolo:epjst07}, MAGNEX riesce ad offrire un'ottima risoluzione in energia, angolo e massa in un angolo solido molto grande.





\clearpage 

\subsection{\iflanguage{italian}{Il rivelatore di piano focale}{The Focal Plane Detector}}

