

%L'apparato sperimentale attualmente in uso ai LNS-INFN nell'ambito della Fase~2 del progetto NUMEN, costituito principalmente dallo spettrometro MAGNEX e dal Ciclotrone Superconduttore K800, viene brevemente illustrato nella prima parte di questo capitolo.
L'apparato sperimentale attualmente in uso ai LNS-INFN nell'ambito della Fase~2 del progetto NUMEN è costituito principalmente dallo spettrometro MAGNEX e dal Ciclotrone Superconduttore K800.
Poiché la descrizione di tale apparato non costituisce l'argomento primario del presente lavoro di tesi, nella prima parte del capitolo  vengono discusse soltanto le sue caratteristiche principali, rimandando alla vasta letteratura scritta sull'argomento per informazioni più dettagliate (ad esempio~\cite{cavallaro:epja12, carbone:epja12, cappuzzello:epja16, cunsolo:epjst07}).

%Nella prima parte di questo capitolo vengono illustrate le principali caratteristiche dell'apparato sperimentale attualmente in uso ai LNS-INFN nell'ambito della Fase~2 del progetto NUMEN.
Nella seconda parte viene descritta la configurazione dell'apparato adottata in occasione del test sui telescopi SiC-CsI svolto ad Aprile 2018, sottolineando le differenze rispetto a quella consueta.



\section{\iflanguage{italian}{Lo spettrometro magnetico MAGNEX}{MAGNEX magnetic spectrometer}}

Lo spettrometro magnetico MAGNEX è un dispositivo ottico a grande accettanza costituito da un quadrupolo per la focalizzazione sull'asse verticale, seguito da un dipolo per la dispersione sul piano orizzontale.
Grazie alle sue peculiarità,  MAGNEX riesce ad offrire, in un angolo solido molto grande e in un ampio range energetico, un'ottima risoluzione in energia, angolo e massa.
Ciò lo rende uno strumento ideale per l'analisi di eventi caratterizzati da sezioni d'urto molto basse, come è già stato dimostrato in~\cite{cappuzzello:epja16,pereira:plb12,oliveira:jpg13}.
Inoltre, esso consente di effettuare misure fino a zero gradi, comprendendo, dunque, la regione angolare di massimo interesse per lo studio del DCE.

La caratteristica che rende MAGNEX uno strumento unico è l'implementazione di una innovativa tecnica di ricostruzione delle traiettorie degli ioni, che consente di correggere le inevitabili aberrazioni originate dalla grande accettanza del dispositivo.
Dunque, a differenza di altri spettrometri magnetici, per MAGNEX è importante determinare non soltanto il punto di impatto sul piano focale ma anche la traiettoria completa. Ciò significa che è necessario misurare quattro parametri: una coppia, chiamata $(x_{foc}, y_{foc})$, individua il punto di impatto, l'altra, indicata con $(\theta_{foc}, \phi_{foc})$, è legata alla traiettoria.
Nel paragrafo successivo verrà esplicato in che modo vengono misurati tali parametri.

In Figura~\ref{fig:magnex} è mostrata una foto di MAGNEX, in cui è possibile notare, andando da sinistra verso destra, la camera di scattering, il quadrupolo, il dipolo e il~FPD (\textcolor{red}{Aggiungere le scritte sull'immagine}).

\begin{figure} [!t]
	\centering
	\includegraphics[width=\textwidth, keepaspectratio]{Grafici/magnex.jpg}
	\caption{Lo spettrometro magnetico MAGNEX.} \label{fig:magnex}
\end{figure}



%\clearpage 

\subsection{\iflanguage{italian}{Il rivelatore di piano focale}{The Focal Plane Detector}}

\begin{figure} [!p]
	\centering
	\includegraphics[width=\textwidth, keepaspectratio]{Grafici/fpd.png}
	\caption{Rappresentazione schematica del FPD: a) vista laterale; b) vista dall'alto. Figura tratta da~\cite{cappuzzello:epja18}.} \label{fig:fpd}
\end{figure}

L'attuale FPD di MAGNEX, la cui rappresentazione schematica è riportata in Figura~\ref{fig:fpd}, è un sistema di rivelazione ibrido, costituito da un tracciatore a gas a bassa pressione e da un muro di rivelatori a pad al silicio.
Esso è posizionato a 1.91~m dall'uscita del dipolo e, al fine di minimizzare gli effetti dovuti alle aberrazioni cromatiche\cite{cunsolo:nima01}, è inclinato di 59.2\textdegree{} rispetto ad un piano perpendicolare all'asse ottico.
Una finestra di mylar spessa 1.5~$\mu$m è utilizzata per contenere il gas, solitamente costituito da N35 isobutano, segnando l'ingresso nel volume attivo.




%Il tracciatore a gas è formato da un sistema di sei fili al tungsteno placcati in oro, posti al di sotto di un anodo segmentato in pad. 
%Il tracciatore a gas lavora secondo il principio tipico delle camera a deriva.
%Il tracciatore a gas è essenzialmente una camera a deriva, in cui un sistema misto di fili e pad consente la misura dei quattro parametri necessari.
%Il tracciatore a gas, che consente la ricostruzione tridimensionale della traiettoria degli ioni, è formato da sei fili al tungsteno, indicati con DC\ped{\textit{i}}, e da un anodo segmentato in pad. 
%Al di sopra di ciascun filo si trova una fila di 224 pad
%Il tracciatore a gas è essenzialmente una camera a deriva, in cui un sistema costituito da sei fili (DC\ped{\textit{i}}) e da un anodo segmentato in pad consente la misura dei quattro parametri necessari per la ricostruzione tridimensionale della traccia.
%In particolare, al di sopra di ciascun filo è presente una fila di pad, le quali sono orientate parallelamente all'asse ottico.
Il tracciatore a gas, che consente la ricostruzione tridimensionale della traiettoria degli ioni, è formato da sei fili proporzionali (DC\ped{\textit{i}}) e da un anodo segmentato in sei file di pad, disposti in modo che sopra ogni filo ci sia una fila di pad.
I fili, sfruttando il principio di lavoro delle camere a deriva, danno una misura di sei posizioni verticali ($Y_i$), mentre le pad permettono di determinare sei posizioni orizzontali ($X_i$).
%Una griglia di Frisch è posta al di sotto dei fili, in quanto questi vengono anche utilizzati per misurare l'energia persa dagli ioni nel gas.
Dal che momento che i fili vengono anche utilizzati per misurare l'energia persa dagli ioni nel gas, una griglia di Frisch è posta al di sotto di essi.




Il muro di rivelatori al silicio è formato da 60 pad, organizzate in 20 colonne da 3 rivelatori ciascuna. (\textcolor{red}{Aggiungo qualche dettaglio in più?})
%Ogni pad ha un'area attiva di $70 \times 50$~mm\ap{2} ed è spessa 500~$\mu$m, sufficienti per fermare i prodotti di reazioni nel range energetico di interesse. 
Essi vengono utilizzati per fermare gli ioni, misurandone l'energia residua e producendo il segnale di trigger per l'acquisizione. 

Quando una particella carica, attraversando la finestra di mylar, entra nel volume attivo, produce nel gas coppie elettrone-ione positivo, le quali, sotto l'effetto di un campo elettrico costante, migrano rispettivamente verso la griglia di Frisch e il catodo. 
%mentre gli elettroni diffondono verso la griglia di Frisch, con una velocità che per questi ultimi è di circa $3 - 5 $~cm/$\mu$s. 
%La presenza di un campo elettrico costante provoca 
%La velocità di deriva tipica degli elettroni è di $3 - 5 $~cm/$\mu$s
Dopo aver attraversato la griglia, gli elettroni giungono in prossimità dei fili DC\ped{\textit{i}}, dove, a causa dell'elevato campo elettrico, danno luogo alla moltiplicazione a valanga. 
%La carica prodotta, proporzionale all'energia persa dalla particella nel gas, induce sulle pad una distribuzione di carica, della quale si calcola il baricentro. Questa operazione avviene per le sei file di pad, in modo tale che ai sei baricentri corrispondono sei misure di posizioni orizzantali $X_i$.
Alle tensioni e pressioni utilizzate, la carica secondaria prodotta genera un segnale proporzionale all'energia persa dalla particella nel gas. 
Poiché ciò avviene per ciascuno dei sei fili, si hanno sei segnali di perdita di energia, indicati con~$\Delta E_i$.

La stessa valanga induce sulle pad una distribuzione di carica, di cui si calcola, tramite un apposito software, il centro di gravità. Anche in questo caso l'operazione si ripete per le sei file di pad, così che vengono estratte sei misure di posizioni orizzontali~$X_i$.
A questo punto, effettuando un fit lineare sulle sei posizioni~$X_i$, si ottengono $x_{foc}$ dall'intercetta e $\theta_{foc}$ dal coefficiente angolare. 

Superata la regione del tracciatore, la particella carica arriva al muro dei rivelatori al silicio, dove si ferma producendo un segnale proporzionale alla sua energia residua~$E_{resid}$. 
Lo stesso segnale viene utilizzato per calcolare l'intervallo di tempo impiegato dagli elettroni primari prodotti nel gas per raggiungere i fili DC\ped{\textit{i}}. 
%Dal momento che tale intervallo è proporzionale allo spazio percorso, si ottengono così sei misure di posizioni verticali~$Y_i$, dalle quali, grazie ad un fit lineare, si ricavano $y_{foc}$ e $\phi_{foc}$ in maniera analoga a quanto visto per $x_{foc}$ e $\theta_{foc}$.
Dal momento che la velocità di deriva è costante, l'intervallo di tempo è direttamente proporzionale allo spazio percorso. Si ottengono, dunque, sei misure di posizioni verticali~$Y_i$, dalle quali, grazie ad un fit lineare, si ricavano $y_{foc}$ e $\phi_{foc}$ in maniera analoga a quanto visto per $x_{foc}$ e $\theta_{foc}$.

È bene ricordare che, nelle camere a deriva a fili, la maggior parte del segnale è originata dal moto degli ioni positivi e non degli elettroni.
Di conseguenza, a causa della minore velocità degli ioni, tale tipologia di rivelatori può tipicamente sostenere rate dell'ordine di pochi~kHz. 
Questo aspetto, che costituisce una delle principali limitazioni all'intensità del fascio tollerabile dall'attuale FPD, deve essere superato nell'ottica della Fase~4. 
Nasce da qui l'esigenza di sostituire l'attuale tracciatore con un sistema in grado di lavorare con un rate elevato di ioni pesanti. 




\clearpage
\section{\iflanguage{italian}{Setting sperimentale nel test}{Experimental setting for the test}}


