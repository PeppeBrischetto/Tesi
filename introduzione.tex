
%A causa del suo carattere elusivo, il neutrino è una delle particelle del Modello Standard di cui si conosce meno: in primo luogo, sebbene da quando sono state osservate le sue oscillazioni di sapore è noto che possiede una massa, ma il valore esatto di tale massa è ancora sconosciuto.

%A causa del suo carattere elusivo, il neutrino è una delle particelle più misteriose del Modello Standard: in primo luogo, non conosciamo il valore esatto della sua massa, ma sappiamo soltanto dei limiti superiori. 
%Inoltre, non siamo certi nemmeno della sua natura, poiché esso potrebbe essere una particella di Dirac o una particella di Majorana.
%Inoltre, non siamo certi nemmeno della sua natura, in quanto, essendo l'unico fermione fondamentale neutro, potrebbe coincidere con la propria antiparticella, come supposto da Ettore Majorana.



Nell'ultimo decennio, l'interesse suscitato dal doppio decadimento beta senza neutrini (\doppiobeta) è cresciuto senza soluzione di continuità, come testimoniano gli innumerevoli esperimenti nati per osservarlo per la prima volta; tale fenomeno rappresenta, infatti, uno strumento fondamentale per svelare alcuni dei misteri che circondano una delle particelle più elusive del Modello Standard: il neutrino. 
Il \doppiobeta{} permetterebbe non soltanto di accedere alla scala assoluta della massa del neutrino, ma anche di chiarire la sua natura fondamentale; fino ad oggi, infatti, non è noto se il neutrino è una particella di Dirac o di Majorana. 
Inoltre, dal momento che tale fenomeno viola la conservazione del numero leptonico di famiglia, potrebbe costituire anche la prima evidenza sperimentale di fisica oltre il Modello Standard.

Per estrarre dagli esperimenti sul \doppiobeta{} le informazioni di interesse, è necessario conoscere gli elementi di matrice nucleare (NMEs) del processo di transizione fra lo stato iniziale e quello finale. 
%Tali NMEs sono finora noti soltanto per via teorica, presentando un quadro non privo di ambiguità.
%Tali NMEs sono finora noti soltanto per via teorica, 
%Finora, le informazioni note su tali NMEs derivano soltanto da modelli teorici, i quali presentano tra loro sostanziali discrepanze.
Tali NMEs sono finora noti soltanto per via teorica, presentando un quadro in cui i diversi modelli esibiscono tra loro sostanziali discrepanze.
Allo scopo di risolvere queste ambiguità è nato il progetto NUMEN (\emph{NUclear Matrix Elements for Neutrinoless double beta decay}), il quale propone un nuovo metodo per accedere sperimentalmente ai NMEs.
Il fulcro di tale metodo è la misura di sezioni d'urto di reazioni di doppio scambio di carica (DCE) indotte da ioni pesanti.
%Questi processi sono caratterizzati da sezioni 
Poiché il progetto intende studiare in modo sistematico tutti gli isotopi candidati al \doppiobeta{} ed essendo i processi di DCE caratterizzati da sezioni d'urto estremamente basse, è stato previsto un profondo upgrade delle due principali infrastrutture sperimentali: il Ciclotrone Superconduttore e lo spettrometro magnetico~MAGNEX. 
Alla fine di tale upgrade i fasci di ioni pesanti avranno un'intensità almeno due ordini di grandezza maggiore di quella attuale.
In quest'ottica è fondamentale lo sviluppo di tecnologie di frontiera, in grado di tollerare le intensità di corrente di fascio previste.

Dal momento che il futuro sistema di tracciamento di MAGNEX, a differenza di quello attuale, non potrà fornire informazioni utili all'identificazione dei prodotti di reazione, il progetto prevede la realizzazione di un muro di telescopi $\Delta E - E$ a stato solido, volto alla Particle IDentification (PID). 
Lo scopo principale di questo lavoro di tesi consiste nel valutare se un sistema basato sulla tecnologia SiC-CsI 




