
%A causa del suo carattere elusivo, il neutrino è una delle particelle del Modello Standard di cui si conosce meno: in primo luogo, sebbene da quando sono state osservate le sue oscillazioni di sapore è noto che possiede una massa, ma il valore esatto di tale massa è ancora sconosciuto.

%A causa del suo carattere elusivo, il neutrino è una delle particelle più misteriose del Modello Standard: in primo luogo, non conosciamo il valore esatto della sua massa, ma sappiamo soltanto dei limiti superiori. 
%Inoltre, non siamo certi nemmeno della sua natura, poiché esso potrebbe essere una particella di Dirac o una particella di Majorana.
%Inoltre, non siamo certi nemmeno della sua natura, in quanto, essendo l'unico fermione fondamentale neutro, potrebbe coincidere con la propria antiparticella, come supposto da Ettore Majorana.



Nell'ultimo decennio, l'interesse suscitato dal doppio decadimento beta senza neutrini (\doppiobeta) è cresciuto senza soluzione di continuità, come testimoniano gli innumerevoli esperimenti nati per osservarlo per la prima volta; tale fenomeno rappresenta, infatti, uno strumento fondamentale per svelare alcuni dei misteri che circondano una delle particelle più elusive dell'Universo: il neutrino. 
Il \doppiobeta{} permetterebbe non soltanto di accedere alla scala assoluta della massa del neutrino, ma anche di chiarire la sua natura fondamentale; fino ad oggi, infatti, non è noto se il neutrino è una particella di Dirac o di Majorana. 
Inoltre, dal momento che tale fenomeno viola la conservazione del numero leptonico totale, potrebbe costituire anche la prima evidenza sperimentale di fisica oltre il Modello Standard.

Per estrarre dagli esperimenti sul \doppiobeta{} le informazioni di interesse sul neutrino è necessario conoscere gli elementi di matrice nucleare (Nuclear Matrix Elements, NMEs) del processo di transizione del nucleo dallo stato iniziale a quello finale. 
%Tali NMEs sono finora noti soltanto per via teorica, presentando un quadro non privo di ambiguità.
%Tali NMEs sono finora noti soltanto per via teorica, 
%Finora, le informazioni note su tali NMEs derivano soltanto da modelli teorici, i quali presentano tra loro sostanziali discrepanze.
Tali NMEs sono finora noti soltanto per via teorica, dando luogo ad uno scenario in cui i diversi modelli mostrano tra loro sostanziali discrepanze.
Allo scopo di risolvere queste ambiguità è nato il progetto NUMEN (NUclear Matrix Elements for Neutrinoless double beta decay), il quale propone ai Laboratori Nazionali del Sud (LNS) dell'Istituto Nazionale di Fisica Nucleare (INFN) un nuovo metodo per accedere sperimentalmente ai NMEs.
Il fulcro di tale metodo è la misura di sezioni d'urto di reazioni di doppio scambio di carica (Double Charge Exchange, DCE) indotte da ioni pesanti, le quali presentano diversi aspetti in comune con il \doppiobeta.
%Questi processi sono caratterizzati da sezioni 
%Poiché il progetto intende studiare in modo sistematico tutti gli isotopi candidati al \doppiobeta{} ed essendo i processi di DCE caratterizzati da sezioni d'urto estremamente basse, è stata prevista una grande opera di ristrutturazione delle due principali infrastrutture sperimentali: il Ciclotrone Superconduttore e lo spettrometro magnetico~MAGNEX. 
Essendo i processi di DCE caratterizzati da sezioni d'urto estremamente basse (tipicamente di pochi~nb), nelle prime fasi del progetto è stato possibile analizzare soltanto alcuni dei casi rilevanti, contraddistinti da particolari condizioni favorevoli.
Tuttavia, al fine di raggiungere il suo massimo, e più ambizioso, obiettivo, il progetto intende studiare in modo sistematico tutti gli isotopi candidati al \doppiobeta{}.
Ciò rende necessario l'utilizzo di fasci di ioni con intensità molto più elevate di quelle attualmente disponibili ai LNS.
A tal fine è stata prevista una grande opera di ristrutturazione delle due principali infrastrutture sperimentali: il Ciclotrone Superconduttore e lo spettrometro magnetico~MAGNEX.
Alla fine dell'upgrade i fasci di ioni pesanti avranno un'intensità almeno due ordini di grandezza superiore a quella attuale: in quest'ottica, per il progetto è fondamentale lo sviluppo di tecnologie di frontiera, in grado di tollerare le correnti previste.
Parte integrante di NUMEN è, dunque, un'intensa attività di ricerca e sviluppo, pertinente non soltanto al campo sperimentale ma anche a quello teorico.


L'upgrade di MAGNEX prevede innanzitutto importanti cambiamenti del rivelatore di piano focale (Focal Plane Detector, FPD), che coinvolgeranno sia il sistema di tracciamento a gas sia il muro di rivelatori al silicio; in particolare, dal momento che il futuro tracciatore, a differenza di quello attuale, non potrà fornire informazioni utili all'identificazione dei prodotti di reazione, il progetto prevede la sostituzione del muro di rivelatori al silicio con uno di telescopi $\Delta E - E$ a stato solido, dedicato alla Particle IDentification (PID). 
Le esigenze di alta resistenza alle radiazioni hanno guidato verso la scelta di un primo stadio costituito da un rivelatore sottile (100~$\mu$m) al carburo di silicio (SiC), seguito da un rivelatore di stop (1~cm) allo ioduro di cesio (CsI).

%Lo scopo principale di questo lavoro di tesi consiste nel valutare se tale sistema permette di distinguere in modo efficace gli ioni nella regione di interesse per NUMEN, costituita principalmente da Ossigeno, Fluoro e Neon.
Lo scopo principale di questo lavoro di tesi consiste nello sviluppo e nell'implementazione di un tool di simulazioni per il progetto NUMEN che, per la prima volta, consenta di descrivere la cinematica di reazione, il moto degli eiettili attraverso gli elementi magnetici di MAGNEX e la loro interazione con il muro di telescopi SiC-CsI.
All'interno di tale tool, i software dedicati al calcolo della cinematica e al trasporto ottico dei prodotti di reazione sono già esistenti e ottimizzati, mentre l'applicazione che riproduce il sistema di rivelazione è stata sviluppata nel corso di questo lavoro.
Tale applicazione è stata realizzata utilizzando la piattaforma \geant{}, la quale, attraverso metodi Monte Carlo, permette di studiare l'interazione delle particelle con la materia.
Lo sviluppo di questo tool di simulazioni è estremamente importante per il progetto, in quanto coinvolge aspetti legati sia alla fase di progettazione sia alla fase operativa del sistema di rivelazione.
%In questo lavoro, il tool è stato utilizzato per analizzare le condizioni ottimali di granularità dei telescopi, valutando le capacità di PID per diverse soluzioni.
In questo lavoro, il tool è stato utilizzato per analizzare e ottimizzare le capacità di PID dei telescopi, valutando diverse soluzioni in termini di granularità.
Per ogni caso considerato si è stimata la frazione di eventi degradati, i quali possono costituire una minaccia dal punto di vista della PID poiché nelle matrici $\Delta E - E$ possono andare a collocarsi in regioni pertinenti ad altri ioni. 
%Lo sviluppo di questo tool permette di condurre uno studio accurato per l'ottimizzazione delle specifiche tecniche dei telescopi, quali ad esempio la granularità, 
%Grazie a questo tool di simulazioni è stato possibile condurre uno studio al fine di ottimizzare le capacità di PID del telescopio e la sensibilità di misura in diverse condizioni di granularità.
Nello studio di processi rari come quelli di DCE è, infatti, essenziale avere un'elevata sensibilità di misura delle sezioni d'urto ed un grande potere di reiezione degli eventi di fondo. 
Il lavoro svolto per questa tesi costituisce il primo passo verso lo sviluppo di una simulazione dell'intero apparato sperimentale dopo l'upgrade, che riproduca fedelmente la risposta di questo agli eventi di interesse.
 
%A tal fine è stata implementata sulla piattaforma \geant{} una simulazione basata su metodi Monte Carlo, grazie alla quale è stato possibile analizzare diverse soluzioni in termini di granularità, di spessore del substrato morto e di estensione della cornice parzialmente attiva attorno alla superficie sensibile del rivelatore al SiC.
%%Per ogni caso considerato si è calcolata la frazione di eventi affetti da fenomeni di raccolta di carica incompleta, i quali possono costituire una minaccia dal punto di vista della PID poiché nelle matrici $\Delta E - E$ possono andare a collocarsi in regioni pertinenti ad altre specie atomiche.
%%Un'attenta disamina di tali eventi è cruciale al fine di garantire una bassa percentuale di errore nella PID, la quale è condizione necessaria per lo studio di processi rari come quelli di~DCE.


Nel Capitolo~1 viene illustrato il contesto scientifico che ha portato alla proposta del progetto NUMEN, descrivendo brevemente gli aspetti essenziali del \doppiobeta{} ed evidenziando le somiglianze tra questo e le reazioni di DCE.
%Vengono, inoltre, discussi gli obiettivi e le fasi del progetto, mettendo in luce l'importanza di questo lavoro di tesi nella prospettiva dell'upgrade del 
Vengono, inoltre, discusse le importanti sfide tecnologiche che il progetto si propone di affrontare al fine di raggiungere gli obiettivi fissati, mettendo in luce l'importanza di questo lavoro di tesi nella prospettiva dell'upgrade del FPD di MAGNEX.


Nel Capitolo~2 vengono presentate le principali caratteristiche dell'attuale apparato sperimentale, descrivendone in breve il principio di funzionamento.
Vengono successivamente illustrati il progetto e il principio di funzionamento del futuro FPD, dedicando particolare spazio al muro di telescopi SiC-CsI.
%Dal momento che i rivelatori al SiC rappresentano un'avanguardia nel campo dei rivelatori di particelle, ne vengono discussi gli aspetti più importanti, sottolineando le proprietà di resistenza alle radiazioni che li hanno resi adatti alle esigenze di NUMEN.
Vengono, quindi, discusse le caratteristiche più importanti dei rivelatori al SiC e degli scintillatori allo CsI, sottolineandone le proprietà di resistenza alle radiazioni, che sono state essenziali per la loro scelta nell'ambito del progetto.
Infine, viene descritto il set-up sperimentale adottato in occasione del test beam svolto ad Aprile 2019, che aveva lo scopo di valutare le performance dei primi due prototipi di telescopi SiC-CsI e del primo esemplare di elettronica di front-end per il progetto.

Nel Capitolo~3, dopo aver esposto i tratti generali delle simulazioni Monte Carlo ed averne spiegato i vantaggi nello studio di problemi ad alta complessità, viene introdotta la piattaforma \geant{}, illustrandone sinteticamente la struttura.
Vengono, dunque, descritti gli aspetti più importanti delle simulazioni svolte per questo lavoro, specificando le condizioni assunte sulla geometria, sulla generazione delle particelle primarie e sui modelli dei processi fisici.
%Infine, vengono presentati i risultati ottenuti, evidenziando nelle diverse condizioni considerate la percentuale di nuclei potenzialmente male identificati.
Infine, vengono presentati i risultati ottenuti, analizzando le performance di identificazione degli ioni al variare dei parametri rilevanti per questo studio e suggerendo le condizioni ottimali dal punto di vista della PID.
 
%Nel Capitolo~4 vengono presentati i risultati del test beam, illustrando le matrici $\Delta E - E$ ottenute.
Nel Capitolo~4 vengono presentati i risultati del test beam, illustrando le correlazioni $\Delta E - E$ ottenute nelle due configurazioni elettroniche.
Viene descritta l'analisi svolta allo scopo di determinare dei parametri di fondamentale importanza per poter simulare le condizioni sperimentali del test.
Infine, viene mostrato il confronto tra i dati sperimentali e quelli simulati, in modo da verificarne la compatibilità.

